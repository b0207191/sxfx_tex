\documentclass[12pt,a4paper]{article}
\usepackage{fontspec}
\usepackage{amsmath}
\usepackage{amssymb}
\usepackage{bm}
\setmainfont{Adobe Kaiti Std}
\thispagestyle{empty}
\pagestyle{empty}
\begin{document}

\newcommand{\ntinf}{\lim\limits_{n \to \infty}}
\newcommand{\ntx}[1]{\lim\limits_{n \to #1}}
\newcommand{\xtx}[1]{\lim\limits_{x \to #1}}

第二章 极限与连续初论

2.1.数列极限

一。概念
1)定义p.29.

$\exists a \in R , \forall \varepsilon > 0, \exists N \in \bm{N}:|x_n-a|<\varepsilon, \forall n>N $

ex.1.

$$\lim_{n \to 0} \frac{1}{(1+\frac{1}{\sqrt{n}})^n}=0$$

因为$(1+a)^n \ge 1+na$

故$\frac{1}{(1+\frac{1}{\sqrt{n}})^n} \le \frac{1}{1+\frac{n}{\sqrt n}} = \frac {1}{1+\sqrt n}< \frac{1}{\sqrt n}$

对于$\forall \varepsilon >0 要使n> \frac{1}{\varepsilon ^ 2},取N=[\frac {1}{\varepsilon ^ 2}]+1$

ex.2.

$$\lim_{n \to \infty} q^n = 0 (0<q<1)$$

$设\frac{1}{q}= 1+\alpha (\alpha > 0)$

$|q^n-0|= \frac{1}{(1+\alpha)^n} \le \frac{1}{1+n\alpha} < \frac{1}{n\alpha}$

对于$\forall \varepsilon > 0, 取n>\frac{1}{\alpha \varepsilon} .......$

$$\lim_{n \to \infty}n^{\frac{1}{n}}=1,当n\ge 2,令n^{\frac{1}{n}}-1=h_n>0即证\lim_{n \to \infty}h_n=0$$

$n=(1+h_n)^n=1+n\cdot h_n+\frac{n(n-1)}{2!}{h_n}^2+......+{h_n}^n>\frac{n(n-1)}{2}{h_n}^2$

$所以0<h_n<\sqrt{\frac{2}{n-1}} (n \ge 2)$

$对\forall \varepsilon > 0 由\sqrt{\frac{2}{n-1}} < \varepsilon 得 n>\frac{2}{\varepsilon ^2}+1$

$取N=max[1,[\frac{2}{\varepsilon ^ 2}]+1]$

(2)极限$$\lim_{n \to \infty}x_n=a等价命题$$

1’ $\forall \varepsilon>0 ,\exists N \in \bm{N}:|x_n-a|<\varepsilon,\forall n>N$

2’ $\forall \varepsilon(0<\varepsilon<1) ,\exists N \in \bm{N}:|x_n-a|<\varepsilon,\forall n>N$

3’ $\forall \varepsilon>0 ,\exists N \in \bm{N}:|x_n-a|<m\varepsilon,\forall n>N(m>0为常数)$

4’ $\forall \varepsilon>0 ,只有有限个x_n满足|x_n-a|\ge\varepsilon$

$\\$

二.收敛数列定理

唯一性,若{$x_n$}收敛,则极限值必唯一

设$$\lim_{x \to \infty}x_n=a, \lim_{x \to \infty}x_n=b$$

令$\varepsilon_0=\frac{1}{2}(b-a),则a-\varepsilon_0<x_n<a+\varepsilon_0,\forall n>N_1$

$b-\varepsilon_0<x_n<b+\varepsilon_0,\forall n>N_2$

$取N=max(N_1,N_2)$

$x_n>b-\varepsilon_0=\frac{a+b}{2}=a+\varepsilon_0>x_n$,矛盾

定义2,若
$\exists M>0:|x_n|\le M,\forall n \in \bm{N}称{x_n}有界$

定理2,收敛数列必有界

定理3,(保号性)若
$$\lim_{n \to \infty}x_n=a>0,必\exists N \in \bm{N}:x_n>0,\forall n > N$$

推论1。若
$$\lim_{n \to \infty}x_n=a>b,必存在N\in \bm{N}:x_n>b,\forall n>N$$

推论2。若
$$\lim_{n \to \infty}x_n=a,\lim_{n \to \infty}y_n=b,a>b,必存在N\in \bm{N}:x_n>y_n,\forall n>N$$

定理4,四则运算法则,前提(极限存在,有限次运算)

若$$\lim_{n \to \infty}x_n=a,\forall \left\{ x_{nk} \right\} \subset \left\{ x_n \right\},\lim_{n \to \infty}x_{nk}=a$$

命题,数列收敛充要条件:奇偶子列均收敛于同一极限

五。数列收敛的条件

(1)迫敛性定理

定理6.设$$\lim_{n \to \infty}y_n=\lim_{n \to \infty}z_n=a,且对\forall n \in N,y_n \le x_n \le z_n,则有$$

$$\lim_{n \to \infty}x_n=a$$

例$(n+1)^\alpha - n^\alpha (0<\alpha<1)=n^\alpha (1+\frac{1}{n})^\alpha - n^\alpha<n^\alpha(1+\frac{1}{n}-1)=\frac{1}{n^{1-\alpha}}$

(2)确界原理(P96确界)

该问题证明应利用定义

ex1. $inf(x_n)+inf(y_n)\le inf(x_n+y_n) \le
\begin{Bmatrix} inf(x_n) & sup(y_n) \\ sup(x_n) & inf(y_n) \end{Bmatrix} \le sup(x_n+y_n) \le sup(x_n)+sup(y_n)
$

(3)单调有界定理 p43 p100

$$x_{n+1}=1+\frac{x_n}{1+x_n},\lim_{n \to \infty}x_{n+1}=1+\frac{\lim\limits_{n \to \infty}x_n}{1+\lim\limits_{n \to \infty}x_n}=\lim_{n \to \infty}x_n$$

$$\lim_{n \to \infty}x_n=\frac{1+\sqrt 5}{2}$$

例题:$设a>0, x_1=a^{\frac{1}{p}} (p \in N), x_{n+1}=\frac{p-1}{p}x_n+\frac{a}{p}x_n^{1-p}$

$$设x_n \ge \sqrt[p]{a} \to  x_{n+1} \le x_n \to \lim_{n \to \infty}x_n = \sqrt[p]{a}$$

柯西收敛原理 p104,等价命题,

$\forall \varepsilon>0, \exists N \in \bm N: |x_{n+p}-x_n|< \varepsilon , \forall n>N, \forall p \in \bm N$

结果须与p无关

$\\$
习题课(数列极限)

一。概念

(1)与$\lim\limits_{n \to \infty}x_n=a等价$

$\exists N \in \bm N, \forall \varepsilon >0:|x_n-a|<\varepsilon, \forall n > N,必有极限$

$但当n>N时,|x_n-a|<\varepsilon,x_n必须为a(从第n+1项起)$

故$\lim\limits_{n \to \infty}x_n=a \nRightarrow 此命题$

$\{x_n\}中有无穷多个子列,趋向于无穷多个极限$

例:

$\includegraphics{sxfx2_p1.jpg}$

趋向于极限1,2,3,。。。。

例2

$\begin{matrix}
1 & \frac{1}{2} & \frac{1}{3} & \frac{1}{4} & ...  \\
\frac{1}{2} & \frac{2}{2} & \frac{3}{2} & \frac{4}{2} & ... \\
\frac{1}{3} & \frac{2}{3} & \frac{3}{3} & \frac{4}{3} & ... \\
\frac{1}{4} & \frac{2}{4} & \frac{3}{4} & \frac{4}{4} & ... \\
...
\end{matrix}$

任意数列存在单调子列$\{x_{nk}\}$

1`$若\{x_n\}中无最大项,任取x_{n1},x_{n2},x_{n3},......构造法易知$

2`$若\{x_n\}中有最大项,记为x_{n1},后面取max\{x_{i>n1}\},记为x_{n2},易知构造法,若后续无最大项,则到情况1$

3`以上,已构造粗一个单调子列,由单调有界定理可知,若有界则有收敛子列


例,设$$\{x_n\}有界,\sup_{n \in \bm N}\{x_n\}=\beta \notin \{x_n\}$$

求证$\{x_n\}$中有严格递增子列,$$\lim_{n \to \infty}x_{nk}=\beta$$

$$
证明:先令\varepsilon_1=1,由确界定义可知\exists x_{n1} \in \{x_n\}:\beta -1 < x_{n1} < \beta
$$

$考虑集合S_1=\{x_n|n>n_1\}仍有sup S_1 = \beta$

$再令\varepsilon_2=min(\frac{1}{2},\beta-x_{n1})$

$必\exists x_{n2} \in S_1 \subset \{x_n\}: \beta- \varepsilon_2<x_{n2}<\beta (n_2>n_1)且x_{n2}>x_{n1}$

$一般的,记S_{k-1}=\{x_n|n>n_{k-1}\},令\varepsilon_k=min(\frac{1}{k},\beta-x_{nk-1}),必$

$\exists x_{nk} \in S_{k-1} \subset \{x_n\},\beta - \varepsilon_k < x_{nk} < \beta, n_k > n_{k-1},且x_{nk} > x_{nk-1} \forall k \in \bm N$

$$因为0<\varepsilon_k \le \frac{1}{k} 故\lim_{n \to \infty}\varepsilon = 0. 令k \to \infty 由迫敛性定理可得 \lim_{n \to \infty}x_{nk}=\beta$$

$且\{x_{nk}\}严格递增$

Ex:

$x_1=\frac{a}{2}, x_n=\frac{a}{2}-\frac{x_{n-1}^2}{2}  (0<a<1)$

解:$x_{2n-1}>x_{2n+1}    x_{2n}<x_{2n+2}$

$x_{2n-1}>x_{2n} \ge x_2, x_{2n}<x_{2n+1}<x_1$

故奇偶子列分别收敛

$$\ntinf x_n = \ntinf (\frac{a}{2}-\frac{x_{n-1}^2}{2}),\ntinf x_{n+1} = \ntinf (\frac{a}{2}-\frac{x_{n}^2}{2})$$

$(\beta = \frac{a}{2}-\frac{\alpha ^2}{2}, \alpha = \frac{a}{2}-\frac{\beta ^2}{2}) \Rightarrow \alpha = \beta$

柯西定理见书P52-P53

柯西第二定理

Ex:$\ntinf \frac {n}{\sqrt[n]{n!}}=\ntinf \frac{x_{n+1}}{x_n}=\ntinf (1+\frac{1}{n})^n=e$

$x_n = \frac{n^n}{n!}$

Ex:$x_n=\sqrt[n]{\begin{matrix}\underbrace{ [a[a...[a]]] } \\ n \end{matrix}},n \in N$

证明$\{x_n\}$收敛

$\\$

函数的极限


一。函数极限概念

(1)过程与极限

一.离散型

二.连续型

1.$x \to + \infty $
2.$x \to - \infty $
3.$x \to \infty $
4.$x \to x_0^- ,x<x_0$
5.$x \to x_0^+ ,x>x_0 $
6.$x \to x_0 $

P64.P65  1`~3`
P63,P55 4`~6`

$当x \to x_0时,f(x)极限不存在$

$\forall A \in R, \exists \varepsilon_0>0, \forall \delta>0:|f(x_a)-A| \ge \varepsilon_0,\exists x_a \in U_0(x_0,\delta)$

$\forall A \in R, \exists \varepsilon_0>0, 及|x_n| (x_n \to x_0,x_n \ne x_0,n \in N),|f(x_n)-A| \ge \varepsilon_0,\forall n \in N$

$例:证明 \ntx{x_0}f(x)=A$

$|f(x)-A| \le ... < c·|x-x_0|,可令|x-x_0|<\delta ,利用。。。。,取\delta =g(\varepsilon)或\delta = min(g(\varepsilon),c`)c`为常数。$

则有$|f(x)-A|<\varepsilon,\forall x \in U_0(x_0,\delta)$

(1)P64

(2)$\ntinf f(x) = A \Leftrightarrow \ntx{+\infty,-\infty} f(x) = A $

(3)Heine定理 P59
$\xtx{x_0}f(x)=A的充要条件是$

$\forall \{x_n\} (x_n \to x_0,x_n \ne x_0) \ntinf f(x_n)=A (可改为\{f(x_n)\}都存在)$

利用:充分性,判断函数极限之性质,必要性,函数极限存在

柯西收敛准则,见P104

$\xtx{x_0}f(x)存在充要条件\forall \varepsilon>0,\exists \delta>0,\forall x',x'' \in U_0(x_0,\delta)有|f(x')-f(x'')|<\varepsilon$

c<b<a,|b|<|a|+|c|

例:
$\xtx{0}(sinx+cosx)^{\frac{1}{x}}= \xtx{0}(1+sinx+cosx-1)^{\frac{1}{sinx+cosx-1}·\frac{sinx+cosx-1}{x}}$

$=e^{\xtx{0} \frac{sinx+cosx-1}{x}}=e^1=e$

$\xtx{0}\frac{ln(1+ax)}{x}=a$

$\xtx{0}\frac{e^{ax}-1}{x}=a$

$\xtx{0}\frac{a^x-1}{x}=lna$

三。数量级及其应用

1)无穷小与无穷大

定义:
$若\xtx{\Delta} f(x)=0,称f(x)为 x \to \Delta 时无穷小 P30$

$\Delta :(x_0,x_0^{+-},\infty,+- \infty)$

性质,在$x \to \Delta 的同一过程中$

(i)有限个无穷小之和在该过程中仍然无穷小

(ii)有界函数与无穷小乘积仍为无穷小

(iii)有限个无穷小乘积仍为无穷小

P46 无穷大
性质P49,P67,P68

Ex:f(x)=x·sinx, $x \to + \infty 无界,但未必f(x) \to \infty$

高阶,P92

Ex:证明 $o(x_n)+o(x_m)=o(x_n)   (m \ge n)$

只需证
$\xtx{0} \frac{o(x_n)+o(x_m)}{x^n}=0$

$\xtx{0} \frac {o(x^n)}{x^n}=0$

$\xtx{0} \frac {o(x^m)}{x^m} \frac{x^m}{x^n} = 0·1 = 0$

无穷小之等价代换(P93)

$x \to 0 可作 \varphi(x) \to 0代换\varphi(x)=x$

sinx~x~tgx ~~~ arcsinx~arctgx~x

$1-cosx ~ \frac{1}{2}x^2,\sqrt[n]{1+x}-1~\frac{x}{n}$

$ln(1+x)~x,e^x-1~x,a^x-1~xlna$

Ex.$f(x)在(0,+\infty)上有f(x^2)=f(x)$

且$\xtx{0+}f(x)=\xtx{+\infty}f(x)=f(1)$

则f(x)=f(1) $\forall x \in (0,+ \infty),当\forall x \in (0,1) f(x)=f(x^2)=....f(x^{2^n})$

$\ntx{\infty}x^{2^n}=0,用归结原理可知f(x)=\ntx{\infty}f(x^{2^n})=...=f(1)$


$x \to \infty,(1-\frac{1}{x^2})^{x^2}=\frac{1}{(1+\frac{1}{-x^2})^{-x^2}}=\frac{1}{e}$

$x \to 0,\frac{a^{x^2}-b^{x^2}}{(a^x-b^x)^2}=\frac{1}{lna-lnb}$

f(x)为三次多项式,且$\xtx{2a}\frac{f(x)}{x-2a}=\xtx{4a}\frac{f(x)}{x-4a}=1,求\xtx{3a}\frac{f(x)}{x-3a}$

设f(x)=b(x-2a)(x-4a)(x-c),代入上式可得

-2ab(2a-c)=1

2ab(4a-c)=1

故有$a^2b=\frac{1}{2},c=3a$

$\xtx{3a}\frac{f(x)}{x-3a}=-\frac{1}{2}$


\end{document}

