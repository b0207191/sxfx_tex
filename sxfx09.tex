\documentclass[12pt,a4paper]{article}
\usepackage{fancyhdr}
\usepackage{fontspec}
\usepackage{amsmath}
\usepackage{amssymb}
\usepackage{bm}
\usepackage{tikz}
\usepackage{pstricks-add}
\setmainfont{Microsoft YaHei}
\pagestyle{fancy}


\begin{document}

\newcommand{\nl}{\newline}
\newcommand{\ntinf}{\lim\limits_{n \to \infty}}
\newcommand{\xtinf}{\lim\limits_{x \to \infty}}
\newcommand{\ntx}[1]{\lim\limits_{n \to #1}}
\newcommand{\xtx}[1]{\lim\limits_{x \to #1}}
\newcommand{\ttx}[1]{\lim\limits_{t \to #1}} 
\newcommand{\ktx}[1]{\lim\limits_{k \to #1}} 
\newcommand{\dxtx}[1]{\lim\limits_{\Delta x \to #1}}
$\nl$

\begin{center} 第9章 广义积分  \end{center}



$\jf{0}{+\infty}\frac{xe^{-x}}{(1+e^{-x})^2}dx=-\frac{x}{1+e^x}|_0^{+\infty}+\jf{0}{+\infty}\frac{dx}{1+e^x}$

$令1+e^x=t,则dx=\frac{1}{t-1}dt,t\in[2,+\infty)$

$故\jf{1}{+\infty}\frac{1}{1+e^x}dx=ln\frac{t-1}{t}|_2^{+\infty}=ln2$

$\nl$

$定理1(Cauchy准则)无穷积分\jf{a}{+\infty}f(x)dx收敛充要条件是\forall \epsilon >0$

$\exists A_0>a:|\jf{A_1}{A_2}f(x)dx|< \epsilon,\forall A_1,A_2 \ge A_0$

$\nl$

$定理2(Heine归结原理)无穷积分\jf{a}{+\infty}f(x)dx收敛充要条件为$

$对\forall \{A_n\} \subset (a,+\infty)且\ntinf A_n=+\infty 极限\ntinf \jf{a}{A_n}f(x)dx存在且相等$

$\nl$

$无穷积分判敛\jf{0}{+\infty}sin^{\frac{1}{3}}xdx \ 分析:用cauchy准则,对  \forall A>0$

$\exists A_1=2n\pi,A_2=(2n+1)\pi >A, 而|\jf{A_1}{A_2}sin^{\frac{1}{3}}xdx|=\jf{2n\pi}{2(n+1)\pi}sin^{\frac{1}{3}}xdx$

$=(令x=(2n\pi+t))\jf{0}{\pi}sin^{\frac{1}{3}}tdt \triangleq C>0,从而发散$

$\nl$

$例:讨论\jf{0}{+\infty}\frac{sin^2x}{x}dx敛散性$

$原式=\sum\limits_{n=1}^{\infty}\jf{n\pi}{(n+1)\pi}\frac{sin^2x}{x}dx,令x=n\pi+t$

$=\sum\limits_{n=0}^{\infty}\jf{0}{\pi}\frac{sin^2t}{t+n\pi}dt \ge \sum\limits_{n=0}^{\infty}\jf{0}{\pi}\frac{sin^2t}{\pi+n\pi}dt = (\frac{1}{\pi}\jf{0}{\pi}sin^2tdt)\sum\limits_{n=0}^{\infty}\frac{1}{n+1}=\frac{1}{2}\sum\limits_{n=1}^{\infty}\frac{1}{n}=+\infty,故发散$

$因为\frac{|sinx|}{x} \ge \frac{sin^2x}{x} 故\frac{|sinx|}{x}发散$

$\nl$

$例:\jf{1}{+\infty}e^{sinx}\frac{sin2x}{x^\lambda}dx(\lambda > 0)$

$可计算|\jf{1}{A}e^{sinxsin2xdx}| \le 2(\forall A>1)$

$而f(x)=\frac{1}{x^\lambda}单调递减(\lambda > 0) 且\to 0$

$当\lambda > 1时,有|e^{sinx}\frac{sin2x}{x^\lambda}| \le \frac{e}{x^\lambda},\forall x>0$

$而\jf{1}{+\infty}\frac{e}{x^\lambda}dx收敛(P-判别法)由此比较可知原式绝对收敛$

$当0 < \lambda \le 1时,又有e^{sinx}\frac{|sin2x|}{x^\lambda} \ge \frac{e^{-1|sin2x|}}{x^\lambda} \ge  \frac{e^{-1}sin^22x}{e^\lambda}$

$=e^{-1}(\frac{1}{2x^\lambda}-\frac{cos4x}{2x^2}) \triangleq I_1(x)-I_2(x)$

$\nl$

$若f(x) \ge 0,g(x)>0,且\xtx{a+}\frac{f(x)}{g(x)}=l$

$l \ne 0,同收敛$

$ l=0,g收敛 \to f收敛 $

$ l=\infty,g发散 \to f发散$

$\nl$

$P判别法(瑕积分),若f(x) \ge 0,且\xtx{a+}(x-a)^Pf(x)=l,① 当0 \le l < +\infty 且$

$P<1时,\jf{a}{b}f(x)dx收敛,②0 < l \le +\infty , P \ge 1, \jf{a}{b}f(x)dx发散$

$\nl$

$例:\jf{0}{1}\frac{lnx}{\sqrt{x}}dx,x=0为瑕点,令P=\frac{3}{4}<1,有\xtx{0+}\frac{lnx}{\sqrt{x}}=0$

$\jf{0}{1}\frac{dx}{\sqrt{x(1-x)}},x=1,x=0为瑕点,故原式=\jf{0}{\frac{1}{2}}\frac{dx}{\sqrt{x(1-x)}}+\jf{\frac{1}{2}}{1}\frac{dx}{\sqrt{x(1-x)}}$

$分别取P=\frac{1}{2},两函数均收敛$

$\jf{0}{1}\frac{x^n}{\sqrt{1-x^4}}dx(n \in Z),当n \ge 0时,x=1为瑕点$

$令P=-n,有\xtx{0+}\frac{1}{\sqrt{1-x^4}}=1,故当-n < 1即n > -1时$

$\jf{0}{\frac{1}{2}}\frac{x^n}{\sqrt{1-x^4}}dx收敛. \forall n \in Z,取P=\frac{1}{2} < 1有$

$\xtx{1}\frac{\sqrt{1-x}x^n}{\sqrt{1-x^4}}=\frac{1}{2},故\forall n \in Z,均有\jf{\frac{1}{2}}{1}\frac{x^n}{\sqrt{1-x^4}}dx收敛$

$\nl$

$例:\jf{0}{+\infty}\frac{x^{\alpha}}{1+x}dx(\alpha \in R),\alpha \ge 0时,令P=1有$

$\xtx{+\infty}x\frac{x^{\alpha}}{1+x}=
\begin{cases} 1, \alpha = 0 \\ +\infty, \alpha > 0 \end{cases}
$

$当\alpha <0时,x=0为瑕点,记为\jf{0}{1}\frac{x^{\alpha}}{1+x}dx+\jf{1}{+\infty}\frac{x^{\alpha}}{1+x}dx,I_1+I_2$

$对I_1(P-判别法) \alpha > -1时收敛,\alpha \le -1 时发散$

$I_2(P法) \alpha > -1时慢收敛,\alpha \le -1 时发散, -1 < \alpha < 0时(当且仅当)收敛$

$\nl$

$Eulor积分,I=\jf{0}{\frac{\pi}{2}}lnsinxdx$

$I=xlnsinx|_0^{\frac{\pi}{2}}-\jf{0}{\frac{\pi}{2}}\frac{x}{tgx}dx,故收敛$

$I=2\jf{0}{\frac{\pi}{4}}lnsin2tdt=\frac{\pi}{2}ln2+2\jf{0}{\frac{\pi}{4}}lnsintdt+2\jf{0}{\frac{\pi}{4}}lncostdt$

$令t=\frac{\pi}{2}-u,又有\jf{0}{\frac{\pi}{4}}lncostdt=\jf{\frac{\pi}{4}}{\frac{\pi}{2}}lnsintdt$

$故有I=\frac{\pi}{2}ln2+2I$

$\nl$

$例:\jf{0}{\pi}xlnsinxdx,令x=\pi-t$

$\jf{0}{1}\frac{lnx}{\sqrt{1-x^2}}dx,令x=sinx$

$\jf{0}{1}\frac{arcsinx}{x}dx,令x=\pi-t$

$\nl$

$Frullani积分,I=\jf{0}{+\infty}\frac{f(ax)-f(bx)}{x}dx,(a,b>0)$

$1^\circ \ f(x)\in C(0,+\infty),\xtinf f(x)存在,则I收敛于[f(0)-\xtinf f(x)]ln\frac{b}{a}$

$2^\circ \ 连续\xtinf f(x)不存在,但\jf{A}{+\infty}\frac{f(x)}{x}收敛(\exists A>0),则I收敛于f(0)ln\frac{b}{a}$

$只证1,令ax=t,bx=t有\jf{a\alpha}{a\beta}\frac{f(t)}{t}dt-\jf{b\alpha}{b\beta}\frac{f(t)}{t}dt$

$=\jf{a\alpha}{b\alpha}\frac{f(t)}{t}dt+\jf{b\alpha}{a\beta}\frac{f(t)}{t}dt-\jf{b\alpha}{b\beta}\frac{f(t)}{t}dt$

$=\jf{a\alpha}{b\alpha}\frac{f(t)}{t}dt+(-\jf{a\beta}{b\beta}\frac{f(t)}{t}dt)$

$=f(\zeta)\jf{a\alpha}{b\alpha}\frac{1}{t}dt-f(\zeta')\jf{a\beta}{b\beta}\frac{f(t)}{t}dt$

$=f(0)ln\frac{b}{a}-ln\frac{b}{a}\xtinf f(x)$

$\nl$

$例:\jf{0}{+\infty}\frac{e^{-ax}-e^{-bx}}{x}dx (情况1),\jf{0}{+\infty}\frac{arctg{(ax)}-arctg{(bx)}}{x}dx(情况1)$

$\jf{0}{+\infty}\frac{cos{ax}-cos{bx}}{x}dx(情况2)$

$\nl$

$积分估计$

$(1)1-\frac{1}{n+1}< \jf{0}{+\infty}e^{-x^n}dx<1+\frac{1}{ne},n \in N$

$\forall x \in R,有e^x \ge 1+x,故当 x \in (0,1)有1-x \le e^{-x} \le 1推出$

$1-x^n \le e^{-x^n} \le 1对上式积分有 1-\frac{1}{n+1} \le \jf{0}{1}e^{-x^n}dx \le 1,而$

$0 < \jf{1}{+\infty}e^{-x^n}dx < \frac{1}{n}\jf{1}{+\infty}nx^{n-1}e^{-x^n}dx=\frac{1}{n}\jf{1}{+\infty}e^{-x^n}dx^n=\frac{1}{ne}$

$两式相加,即有$

$\nl$

$(2)\jf{x}{+\infty}\frac{cost}{\sqrt t}dt+\frac{sinx}{\sqrt x}=o(\frac{1}{x \sqrt x}),x \to +\infty$

$两次分步积分,得\jf{x}{+\infty}\frac{cost}{\sqrt t}dt=-\frac{sinx}{\sqrt x}+\frac{cosx}{2x\sqrt x}-\frac{3}{4}\jf{x}{+\infty}\frac{cost}{t^2\sqrt t}dt(只需证\frac{cosx}{2x\sqrt x}-\frac{3}{4}\jf{x}{+\infty}\frac{cost}{t^2\sqrt t}dt除以(\frac{1}{x\sqrt x})为有界量)$

$可计算\forall A > x,\exists M >0$

$|\frac{\jf{x}{A}\frac{cost}{t^2\sqrt t}dt}{\frac{1}{x\sqrt x}}| = |\frac{cos\zeta\jf{x}{A}\frac{1}{t^2\sqrt t}dt}{\frac{1}{x\sqrt x}}| = ....$

$\nl$

$(3)\jf{1.9}{2}\frac{e^{-x}}{\sqrt[4]{2+x-x^2}}dx < 0.03$

$用Schwaz不等式(\jf{1.9}{2}\frac{e^{-x}}{\sqrt[4]{2+x-x^2}}dx)^2 \le (\jf{1.9}{2}\frac{e^{-x}}{\sqrt{1+x}}dx)(\jf{1.9}{2}\frac{dx}{\sqrt{2-x}})$

$\le (\frac{1}{\sqrt{2.9}}\jf{1.9}{2}e^{-x}dx)(2-\sqrt{2-x})|_2^{1.9}= 0.000753$

$开根号,即<0.027441 $

$另:原式=\jf{1.9}{2}\frac{e^{-x}}{\sqrt[4]{1+x}}\frac{1}{\sqrt[4]{2-x}}dx < \frac{e^{-1.9}}{\sqrt[4]{2.9}}\jf{1.9}{2}\frac{1}{\sqrt[4]{2-x}}dx < 0.03$


$\nl$

$\nl$

$习题课$

$\jf{1}{+\infty}\frac{sinx}{\sqrt x}dx收敛,\jf{1}{+\infty}\frac{sin^2x}{x}dx发散$

$\nl$

$\jf{a}{b}f^2(x)dx(a为瑕点) \Leftarrow \jf{a}{b}|f(x)|dx$

$证明0 \le |f(x)| \le \frac{1}{2}[1+f^2(x)] \forall x \in (a,b),但对无穷积分不成立$

$\jf{a}{b}f^2(x)dx(a为瑕点) \nRightarrow \jf{a}{b}|f(x)|dx,反例:\jf{0}{1}\frac{1}{\sqrt x}dx$

$\nl$

$命题:f(x)满足下列条件之一,则当\jf{a}{+\infty}f(x)dx收敛时,必有\xtinf f(x)=0$

$1:\xtinf f(x)存在,2:单调,3:一致连续。$

$反例:\jf{0}{+\infty}(-1)^{[x^2]}dx,\jf{0}{+\infty}sin(x^2)dx$

$\nl$

$\jf{a}{+\infty}f(x)dx绝对收敛,g(x)有界,\jf{a}{+\infty}f(x)g(x)dx收敛?$

$由0 \le |f(x)g(x)| \le M|f(x)|可知其绝对收敛$

$\nl$

$若f,g \in C(a,+\infty)且\jf{a}{+\infty}f(x)dx收敛,\xtinf g(x)=1,是否\jf{a}{+\infty}f(x)g(x)dx收敛?$

$反例:f=\frac{sinx}{\sqrt x},g=(1+f)$

$\nl$

$若\exists p>1,使得f(x)=o(\frac{1}{x^p})(x \to +\infty),则\jf{a}{+\infty}|f(x)|dx收敛$

$\nl$

$对瑕积分,P<1时收敛,P>1时发散$

$若f(x)~\frac{c}{x^p}(c \ne 0,x \to +\infty)则当p>1时,\jf{a}{+\infty}f(x)dx收敛,当p \le 1时,\jf{a}{+\infty}f(x)dx发散$

$\nl$
$例:\frac{1}{\sqrt x (1+x^2)} ~ \frac{1}{x^{\frac{5}{2}}},x \to +\infty $

$\frac{1}{\sqrt x (1+x^2)} ~ \frac{1}{x^{\frac{1}{2}}},p=\frac{1}{2}<1$

$故\jf{0}{+\infty}f(x)dx收敛$

$\nl$

$\jf{0}{+\infty}[ln(1+\frac{1}{x})-\frac{1}{1+x}]dx$

$x \to +\infty, ln(1+\frac{1}{x})=\frac{1}{x}-\frac{1}{2x^2}+o(\frac{1}{x^2})$

$\frac{1}{1+x}=\frac{1}{x}\frac{1}{1+\frac{1}{x}}= \frac{1}{x}[1-\frac{1}{x}+o(\frac{1}{x})]$

$ln(1+\frac{1}{x})-\frac{1}{1+x}=\frac{1}{2x^2}+o(\frac{1}{x^2})~\frac{1}{2x^2},P=2,故发散$

$令x=\frac{1}{t},\jf{0}{1}ln(1+\frac{1}{x})dx=\jf{1}{+\infty}\frac{ln(1+t)}{t^2}dt$

$\nl$

$设f(x)在任意有限区间上可积,\xtinf f(x)=C证明:$

$\xtx{0+}\lambda \jf{0}{+\infty}e^{-\lambda x}f(x)dx=C$

$分析:C=\lambda \jf{0}{+\infty}Ce^{-\lambda x}dx,故只须证明|\lambda \jf{0}{+\infty}e^{-\lambda x}(f(x)-C)dx|$

$\forall \epsilon > 0, \exists A>0,|f(x)-C|< \frac{\epsilon}{2},\forall x \ge A,故\jf{0}{A}|f(x)-C|dx \triangleq M$

$|\lambda \jf{0}{+\infty}e^{-\lambda x}f(x)dx-C| \le \lambda |\jf{0}{A}e^{-\lambda}|f(x)-C|dx|+\lambda \jf{A}{+\infty}e^{-\lambda x}|f(x)-C|dx$

$\le \lambda M+\frac{\epsilon}{2} \jf{A}{+\infty}e^{-\lambda x}d(\lambda x) < \lambda M+ \frac{\epsilon ll}{2}$

$取\delta (0 < \delta < \frac{\epsilon}{2M}),则当0<\lambda<\delta 时,有\lambda M+\frac{\epsilon}{2} < \epsilon,即有\xtx{0+}\lambda \jf{0}{+\infty}e^{-\lambda x}f(x)dx=C$

$\nl$

$例:设g(x)在[0,+\infty)上非负,\Atinf \jf{0}{A} g(x)dx=\alpha,f(x) \in c[0,1]$

$证明: \ttx{o^+}\jf{0}{1}t^{-1}g(t^{-1}x)f(x)dx=\alpha f(0)$

$依题有|f(x)| \le M,\forall x \in [0,1],由f(x)在x=0的连续,\forall \epsilon >0, \exists \delta >0:$

$|f(x)-f(0)|<\epsilon, \forall x \in \cup (0,\delta)$

$对上述\epsilon >0,\delta >0, \exists t_0 > 0,当\forall t(0<t<t_0)由Cauchy准则有$

$0=\jf{\frac{\delta}{t}}{\frac{1}{t}}g(u)du < \epsilon$

$注意到\ttx{0^+}\jf{0}{1}t^{-1}g(t^{-1}x)dx=$

$\ttx{0^+}\jf{0}{\frac{1}{t}}g(u)du=\alpha,此时有|\jf{0}{1}t^{-1}g(tx)|f(x)-f(0)|dx| \le$

$\jf{0}{\delta}t^{-1}g(t^{-1}x)|f(x)-f(0)|dx+\jf{\delta}{1}t^{-1}g(t^{-1}x)|f(x)-f(0)|dx$

$< \epsilon \jf{0}{\frac{\epsilon}{t}}g(u)du+2M|\jf{\frac{\epsilon}{t}}{\frac{1}{t}}g(u)du| < \epsilon(\alpha+2M)$

$故原式=f(0)\ttx{0}\jf{0}{1}t^{-1}g(t^{-1}x)dx=f(0)\ttx{0}\jf{0}{\frac{1}{t}}g(u)du=\alpha f(0)$

$证明方向:\epsilon \to \delta \to t_0(0 < t < t_0)$

\end{document}

