\documentclass[12pt,a4paper]{article}
\usepackage{fancyhdr}
\usepackage{fontspec}
\usepackage{amsmath}
\usepackage{amssymb}
\usepackage{bm}
\usepackage{tikz}
\usepackage{pstricks-add}
\setmainfont{Microsoft YaHei}
\pagestyle{fancy}


\begin{document}

\fancyfoot[C]{by chinasjtu@msn.com }

\newcommand{\nl}{\newline}

\newcommand{\ntinf}{\lim\limits_{n \to \infty}}
\newcommand{\xtinf}{\lim\limits_{x \to \infty}}

\newcommand{\Atinf}{\lim\limits_{A \to \infty}}
\newcommand{\Rtinf}{\lim\limits_{R \to \infty}}

\newcommand{\ntx}[1]{\lim\limits_{n \to #1}}
\newcommand{\xtx}[1]{\lim\limits_{x \to #1}}
\newcommand{\ttx}[1]{\lim\limits_{t \to #1}} 
\newcommand{\ktx}[1]{\lim\limits_{k \to #1}} 
\newcommand{\dxtx}[1]{\lim\limits_{\Delta x \to #1}}

\newcommand{\jfab}{\int_{a}^{b}}
\newcommand{\jf}[2]{\int_{#1}^{#2}}

\newcommand{\nsum}[2]{\sum\limits_{n=#1}^{#2}}
\newcommand{\isum}[2]{\sum\limits_{i=#1}^{#2}}
\newcommand{\ksum}[2]{\sum\limits_{k=#1}^{#2}}

\newcommand{\nsuminf} {\nsum{1}{\infty}}
\newcommand{\ksuminf} {\ksum{1}{\infty}}
\newcommand{\isuminf} {\isum{1}{\infty}}





\begin{center} 第一章 函数概论  \end{center}
$\nl$
$\S$1. 函数概念

$\nl$
一.变量函数


所有不超过正实数x的质数个数 N=$\Pi$(x) ~~  $\Pi(3000)=430$

$\nl$
定义1 设X $\subset$ R  ~~ Y $\subset$ R 为两数集,若存在X到Y的映射f

f:X $\to$Y ~~ x$\to$y=f(x)

称f为定义在X上的(一元单值)函数

$注1^\circ  f定义域为X的值域$

\begin{center}$ f(X)=\{y|y=f(x),x\in X\}\subset Y   $ \end{center}

$2^\circ$ 函数相等$\iff$有相同定义域X且f(x)=g(x)  x$\in$X

但映射方法也许不同,比如f=x,  g=$\sqrt{x^2}$ ~(x$\ge$0)

$\nl$
二.特殊函数

1.符号函数  

$
 y=sgn x= \begin{cases} 
-1, & x < 0 \\
0, & x = 0 \\
1, &  x > 0 
\end{cases}
$

$\nl$
2. 方括号函数 

y= [x] = n (n$\le$x<n+1)

性质:[x]$\le$x<[x]+1

[x+n]=[x]+n

$\nl$
3.Dirichlet函数 

$
D(x)=\begin{cases}
1, & x\in Q \\
0, & x\notin Q
\end{cases}
D(x) \in R
$

$\nl$
4.定义在[0,1] ~ Riemann函数 

$
R(x)=\begin{cases}
1/q, & x=p/q (p,q互质,p<q) \\
1, & x=0,1 \\
0, & x为[0,1]上无理数
\end{cases}
$

$\nl$
三.复合函数

定义2 设函数y=f(x) x$\in$X,y$\in$Y  
z=g(y)  y$\in Y'$ 

若Y$\subset Y'则可在X上定义z为x的复合函数$

z=g(f(x))  x$\in$X (注$Y \bigcap Y' \ne  \emptyset $  记为 $\bar{Y}$) 

$\bar{X}= \{ x|f(x)\in\bar{Y} \} \subset X $

$f(f(x))=(f \cdot f)(x)$

例:
$
f(x)=\begin{cases}
1+x, & x < 0 \\
0, & x \ge 0
\end{cases}
$

$
f(f(x))=\begin{cases}
2+x, & x < -1 \\
1, & x \ge -1
\end{cases}
$

$
f^n(x)=\begin{cases}
n+x, & x < 1-n \\
0, & x \ge 1-n
\end{cases}
$

$\nl$
四.反函数

定义3,若f:X$\rightarrow$Y=f(X) x|$\rightarrow$y=f(x)

是X到f(x)的一一映射时,则称f的逆映射$f^{-1}$

$f^{-1}:f(x)\rightarrow X ~~  y\rightarrow X=f^{-1}(y)$

为y=f(x)的反函数 ~~ 记为$x=f^{-1}(y)  ~~ y \in f(X)$

$f^{-1}\cdot f=I : X \rightarrow X$  ~~
$f \cdot f^{-1}=I : Y \rightarrow Y$  

$(f^{-1})^{-1}=f$

反函数存在关键:一一映射

$\nl$
五.初等函数
p20-25 反对幂指三(分部积分顺序)

$\\$
$\nl$
$\S 1.2具有某些特征的函数$
$\nl$
一.有界函数,无界函数(须在区间内讨论)

定义1 函数f(x)~~x$\in$X有界,是指
$\exists M>0 ~~~ f(x)|\le M ~~ \forall x \in X$

若$\exists m \in R $ ~~ f(x)$\le$m ~~ $\forall x \in X$ 有上界

若$\exists m \in R $ ~~ f(x)$\ge$m ~~ $\forall x \in X$ 有下界

$\nl$
二.单调函数

定义2 设函数f(x) ~ x$\in$X ~ 若$\forall x_1 ~ x_2 \in X (x_1<x_2)$

$f(x_1) \le f(x_2)  (或f(x_1) \ge f(x_2))$
称f(x)在X上递增(或递减)


定理1. 设y=f(x) ~ x$\in$X 若f(x)在X上严格单调,

则f的反函数$f^{-1}$存在且x=$f^{-1}$(y)也是严格单调的

凡严格单调,具有反函数

奇函数,偶函数

$\nl$
三.周期函数


定义3,设$f(x)~x\in X 若\exists \omega \ne 0 对 \forall x \in X$

$x\pm \omega \in X 且 f(x\pm \omega) = f(x)$

称f(x)为以$\omega$为周期的周期函数 Z(x)

例4 ~ 证明 f(x)=cos$\sqrt x$不是 Z(x)

反证法,若$\exists \omega >0, cos\sqrt{(x+\omega )} = cos\sqrt x$ 和差化积

则$sin(\frac{\sqrt{x+\omega}}{2}+\frac{\sqrt{x}}{2})sin(\frac{\sqrt{x+\omega}}{2}-\frac{\sqrt{x}}{2})=0$ ~~~ $\forall x \in R$

令x=0,有sin$\frac{\sqrt{\omega}}{2}=0,故\omega = 4k^2\pi^2,令x=4\pi^2$

有$sin(\sqrt{k^2+1}+1)\pi=0,即\sqrt{k^2+1}+1=m,m\in N,即k^2+1=(m-1)^2$

故k=0

$\nl$
习题课1

$f(x),x\in(a,b) 若f(x)在\forall (\alpha,\beta]\subset(a,b)有界,不一定f(x)在(a,b)上有界,例f(x)=1/x,f(x)=tgx$

$f(x)=0 周期函数,奇函数,单调函数$

$f,g均为X上单调增函数$

$f+g必为X上增函数?成立$

$f×g必为X上增函数?不一定,f(x)=g(x)=x$

$f\cdot g必为X上增函数? 成立$

$f,g为X上有界 f/g不一定有界,f=1,g=x,X=[-1,1]$

$若f(g)=g(f)是否有f=g,反例:f=x g=x^3$


$\\$

构造f(x),$x\in[0,1]$,f(x)在[0,1]上处处有极限,但在[0,1]任意点任意邻域内均无界
$
f(x)=\begin{cases}
q, & 当x=\frac{p}{q}时,q若有限,则p也有限 \\
0, & 其余
\end{cases}
$

$\nl$

不等式

$1 \cdot (1+\frac{1}{n})^n \le [\frac{1+n(1+\frac{1}{n})}{n+1}]^{n+1} (AG不等式)$

$=(1+\frac{1}{n+1})^{n+1}$


证明:
$\frac{1}{2\sqrt{n}} < \frac{(2n-1)!!}{(2n)!!} < \frac{1}{\sqrt{2n+1}} 数学归纳法证$


$\nl$

非Z(x)的证明方法

1)f(x)=0的零点非周期

2)反证法


例 f(x)=sin($x^2$)

1)$x_{n+1}-x_n$无限减小

$2)令x=0,利用和差化积得\omega^2=k\pi,后令x=\sqrt2 \omega,有sin(1+\sqrt 2)^2k\pi=0$推出矛盾

\end{document}

