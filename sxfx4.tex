\documentclass[12pt,a4paper]{article}
\usepackage{fontspec}
\usepackage{amsmath}
\usepackage{amssymb}
\usepackage{bm}
\setmainfont{Adobe Kaiti Std}
\thispagestyle{empty}
\pagestyle{empty}
\begin{document}

\newcommand{\nl}{\newline}
\newcommand{\ntinf}{\lim\limits_{n \to \infty}}
\newcommand{\ntx}[1]{\lim\limits_{n \to #1}}
\newcommand{\xtx}[1]{\lim\limits_{x \to #1}}
\newcommand{\ttx}[1]{\lim\limits_{t \to #1}}


第4章 导数与微分

$注1':f'(x_0)=\xtx{x_0}\frac{f(x)-f(x_0)}{x-x_0}$

$2',\Delta y=f'(x_0)\Delta x+o(\Delta x)可导必连续$

设f(x)在x=a处可导

$\ttx{0}\frac{f(a+t)-f(a-2t)}{2t}=\ttx{0}\frac{f(a+t)-f(a)-[f(a-2t)-f(a)]}{2t}$

$=\ttx{0}\frac{f(a+t)-f(a)}{2t}+\ttx{0}\frac{f(a-2t)-f(a)}{-2t}$

$=\frac{3}{2}f'(a) 逆命题未必成立$

$例:设f(x)=|x-a|\varphi(x),其中\varphi(x)为连续函数,求f'-(a),f'+(a)$

问f(x)在x=a处可导条件

$\frac{\Delta y}{\Delta x}=\frac{f(a+\Delta x)-f(a)}{\Delta x} = \frac{|\Delta x|\varphi(a+\Delta x)}{\Delta x}$

$=\pm \varphi(a+\Delta x),\Delta x \lessgtr 0$

$f'-(a) = -\varphi(a),f'+(a) = +\varphi(a),当仅当\varphi(a)=0时可导$

定理2:
$设f(x)在x_0处左右导数均存在,则f(x_0)在x_0处必连续$

证明:
$\lim\limits_{\Delta x \to 0-}[f(x_0+\Delta x)-f(x_0)]=\lim\limits_{\Delta x \to 0-}\frac{[f(x_0+\Delta x)-f(x_0)]}{\Delta x} \Delta x$

$=f'-(x_0)·0=0,故左连续,同理可证,右连续$

复合函数(链式法则),对数法则

$ex.1. y=\frac{(x+5)^2}{(x+2)^3}$

$lny=2ln(x+5)-3ln(x+2)$

$\frac{y'}{y}=\frac{2}{x+5}-\frac{3}{x+2}$

$y'=\frac{(x+5)^2}{(x+2)^3}(\frac{2}{x+5}-\frac{3}{x+2})$

$ex.2.y=x+x^x+x^{x^x}$

$y'=1+x^x(1+lnx)+...$

$\lim\limits_{\Delta x \to 0}\frac{\Delta y}{\Delta x}=\frac{dy}{dx},\frac{\Delta y}{\Delta x}=\frac{dy}{dx}+o(x)$

$\\$
$4.3.微分$

一,微分概念

$定义:y=f(x),x\in U(x_0),若\exists A \in R$

$\Delta y = f(x_0+\Delta x)-f(x_0)=A·\Delta x+o(\Delta x)$

$称f(x)在x_0可微,A\Delta x为f(x)在x_0处微分,记为dy|_{x=x_0},df|_{x=x_0}$

几何意义,在$(x_0,y_0)$处切线纵坐标改变量

$\includegraphics{sxfx4_p1.jpg}$

高阶导数,高阶微分,P162

$dx^k=(dx)^k,d^kx表示对x的k次微分$

$f(x)在x_0处n阶可导 \to f^{n-1}(x)在x_0处连续 \to f^{n-k}(x) \in C(U(x_0)),k=2,3,4...n$

Ex.
$y=\frac{lnx}{x},y^n=\frac{(-1)^nn!}{x^{n+1}}[lnx-(1+\frac{1}{2}+\frac{1}{3}+...\frac{1}{n})]$

分段点处求导用定义

Ex.$y=arctg(x),求y^n(x)$

$y^1(-x^2+1)=1,同时求n-1阶导数$

$y'=\frac{1}{2i}(\frac{1}{x-i}-\frac{1}{x+i})$

$c_{n-1}^0(y')^{(n-1)}(1+x^2)^{(0)}+c_{n-1}^1(y')^{(n-2)}(1+x^2)^{(1)}+c_{n-1}^2(y')^{(n-3)}(1+x^2)^{(2)}$

$y^{(n)}(1+x^2)+(n-1)y^{(n-1)}2x+c_{n-1}^2y^{(n-2)}2=0$

$x=0代入,y_{(0)}^{(n)}=-(n-1)(n-2)y^{(n-2)}(0) \gets 递推公式$

$n=2k时,f^{(2k)}(0)=0$

$n=2k+1时,f^{(2k+1)}(0)=(2k)!(-1)^k$

隐函数P152

x=a(cost+t·sint)

y=a(sint-t·cost)

$\frac{dy}{dx}=tgt$

极坐标

$r=r(\theta)$

$x=r(\theta)cos\theta$

$y=r(\theta)sin\theta$

$\frac{dy}{dx}=\frac{r'(\theta)tg\theta+r(\theta)}{r'(\theta)-r'(\theta)tg\theta}$

$\newline$

设ABC为平面三角形,l为已知直线,证明必存在平行l直线平分三角形面积

根据l方向建立坐标系(介值定理)

$\includegraphics{sxfx4_p2.jpg}$

$A,B,C,(x_1,x_2,x_3)$

$x_1>x_2<x_3$

定义左侧面积为S(x)

$|S(x)-S(x'')| \le |x''-x|·max(y_1,y_2,y_3)$

故S(x)为一致连续函数

$\nl$

平面上有n个点,证明必存在一面积最小的圆,将此n个点全部盖住,且此圆唯一

证明:唯一性(易证)。

$\includegraphics{sxfx4_p3.jpg}$

有R,则有r<R,矛盾。

存在性,确界定理。

设f(x)在[a,b]上定义,若f(x)每个值恰好取两次,证明f(x)在[a,b]不连续

假设连续,则可取最大值,最小值2次

$\includegraphics{sxfx4_p4.jpg}$

$设f(x_3)>f(x_1),f(x_3)>f(x_2)$

$则x_1与A,A与x_2之间有与f(x_3)相同的值存在,矛盾$

$\nl$

设f(x)在R上一致连续,证明必存在A,B>0,使得|f(x)|$\le$A|x|+B

证明:$取\varepsilon_0=1,由f(x)\in U(R),故有\exists \delta >0,|f(x_1)-f(x_2)|<\varepsilon_0=1,\forall x_1,x_2 \in R,|x_1-x_2|<\delta$

$\forall x \in R,|x|\le \delta,取n=[\frac{|x|}{\delta}]+1$

$由\frac{|x|}{\delta}<n\le\frac{|x|}{\delta}+1 \to \frac{|x|}{n}<\delta\le\frac{|x|}{n-1}$

$于是|f(x)-f(0)|\le|f(x)-f(\frac{n-1}{n}x)|+|f(\frac{n-1}{n}x)-f(\frac{n-2}{n}x)|+...+|f(\frac{1}{n}x)-f(0)|<n·\varepsilon_0=n$

故有$|f(x)|<|f(0)|+n\le|f(0)|+\frac{|x|}{\delta}+1$

$若|x|<\delta,|f(x)|<|f(0)|+1\le|f(0)|+\frac{|x|}{\delta}+1$

取$A=\frac{1}{\delta},B=|f(0)|+1,\delta 是对于\varepsilon_0 取1时而有的 $

$\nl$

$\includegraphics{sxfx4_p5.jpg}$

无界数列中必有无穷大子列

隐函数与参数方程的高阶导数P168

$\nl$

$f(u)在u_0可导,u=g(x)在x_0不可导,f(g(x))在x_0处是否必不可导,反例u=|x|,f(u)=u^2$

$f(u)在u_0不可导,u=g(x)在x_0不可导,f(g(x))在x_0处是否必不可导,反例f=g=\frac{1}{x}$

$f(x)\ge g(x) 必有f'(x)\ge g'(x)? 反例:f=e^{-x},g=-e^{-x}$

$\nl$

$(\frac{1-x}{1+x})^{(n)}=\frac{2(-1)^nn!}{(1+x)^{n+1}}$

$f(x)在(a,b)内可导,\xtx{a+0}f(x)=\infty,是否必有\xtx{a+0}f'(x)=\infty$

反例$f(x)=\frac{1}{x}+cos\frac{1}{x},x\in(0,1)$

逆命题反例$y=\sqrt{x}$

$f(x)在[a,+\infty)上可导,\xtx{+\infty}f(x)存在,\xtx{+\infty}f'(x)也存在?$

反例$f(x)=\frac{sin(x^2)}{x}, x\in[1,+\infty]$

f(x)为奇偶性具有的函数,f'(x)也具有奇偶性,但奇偶性改变

f(x)为周期性具有的函数,f'(x)也具有周期性

$y=|x+1|^3-2,x \in R,求f'(x),f''(x)$

$
y'=\begin{cases}
3(x+1)^2, & x>-1 \\
0, & x=-1 \\
-3(x+1)^2, & x<-1
\end{cases}
$

$
y''=\begin{cases}
6(x+1), & x>-1 \\
0, & x=-1 \\
-6(x+1), & x<-1
\end{cases}
$

$f''(-1)用f'(x)的x=-1的左右导数来求,f''_{\pm}(-1)=0$

$f'''_{\pm}(-1)=\xtx{0\pm}\frac{\pm 6 x-0}{x}=\pm 6$

$/nl$

$
f(x)=\begin{cases}
x^2e^{-x^2}, & |x|\le 1 \\
\frac{1}{e}, & |x|>1
\end{cases}
$

$
f'(x)=\begin{cases}
2xe^{-x^2}(1-x^2), & |x|\le 1 \\
0, & |x|>1
\end{cases}
$

$f'_+(1)=0,f'_-(0)=\xtx{1-}\frac{x^2(e^{1-x^2}-1)+x^2-1}{e(x-1)}$

$=\xtx{1-}[-x^2(1+x)\frac{e^{1-x^2}-1}{e(1-x^2)}+\frac{1+x}{e}]$

$=-\frac{2}{e}+\frac{2}{e}=0$

$\nl$
$例,设f(x)在[a,b]上可导,且f'_+(a) \ne f'_-(b),则对\forall C$

$C介于f'_+(a)和f'_-(b)之间,则\exists x_0 \in (a,b),有$

$f'(x_0)=C \Leftrightarrow \exists (f(x)-Cx)'=0$

$不妨设f'_+(a) < C < f'_-(b),构造F(x)=f(x)-Cx$

$则F(x)在[a,b]上可导,F(x)\in C[a,b]$

F(x)在[a,b]上有最大值和最小值

$记F(x_0)=minF(x)$
由Fermat引理有$F'(x_0)=0 \to f'(x_0)=C$

$若x_0=a或b,则F(a) \le F(x),\forall x \in U(a)$

则有$\frac{F(x)-F(a)}{x-a} \ge 0,F'(a)=f'_+(a)-C \ge 0$

$f'_+(a)\ge C 与条件矛盾$

\end{document}

