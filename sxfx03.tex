\documentclass[12pt,a4paper]{article}
\usepackage{fancyhdr}
\usepackage{fontspec}
\usepackage{amsmath}
\usepackage{amssymb}
\usepackage{bm}
\usepackage{tikz}
\usepackage{pstricks-add}
\setmainfont{Microsoft YaHei}
\pagestyle{fancy}


\begin{document}

\fancyfoot[C]{by chinasjtu@msn.com }

\newcommand{\nl}{\newline}

\newcommand{\ntinf}{\lim\limits_{n \to \infty}}
\newcommand{\xtinf}{\lim\limits_{x \to \infty}}

\newcommand{\Atinf}{\lim\limits_{A \to \infty}}
\newcommand{\Rtinf}{\lim\limits_{R \to \infty}}

\newcommand{\ntx}[1]{\lim\limits_{n \to #1}}
\newcommand{\xtx}[1]{\lim\limits_{x \to #1}}
\newcommand{\ttx}[1]{\lim\limits_{t \to #1}} 
\newcommand{\ktx}[1]{\lim\limits_{k \to #1}} 
\newcommand{\dxtx}[1]{\lim\limits_{\Delta x \to #1}}

\newcommand{\jfab}{\int_{a}^{b}}
\newcommand{\jf}[2]{\int_{#1}^{#2}}

\newcommand{\nsum}[2]{\sum\limits_{n=#1}^{#2}}
\newcommand{\isum}[2]{\sum\limits_{i=#1}^{#2}}
\newcommand{\ksum}[2]{\sum\limits_{k=#1}^{#2}}

\newcommand{\nsuminf} {\nsum{1}{\infty}}
\newcommand{\ksuminf} {\ksum{1}{\infty}}
\newcommand{\isuminf} {\isum{1}{\infty}}





\begin{center} 第三章 连续函数  \end{center}

P76

$设y=f(x),x \in U(x_0) ~ 若x在x_0处改变量为\Delta x,则y的改变量\Delta y=f(x_0+\Delta x)-f(x_0)满足$

$\dxtx{0} \Delta y = \dxtx{0}[f(x_0+\Delta x)-f(x_0)]=0,称为f(x)在x_0连续$

$\nl$
p.s.极限证明问题

设$x_1=a,x_2=b,x_{n+1}=\frac{x_n+x_{n-1}}{2},x_n极限存在$

证明:$\frac{x_{n+1}-x_n}{x_n-x_{n-1}}<0有单调性(对于奇偶子列)$

$\nl$

设$\ntinf \frac{a_1+a_2+...+a_n}{n}=a,证明:\ntinf \frac{a_n}{n}=0$

用Cauchy准则,由条件$\forall \varepsilon >0,\exists N_1 \in N$

$|\frac{a_1+a_2+...+a_n}{n}-\frac{a_1+a_2+...+a_n+a_{n+1}}{n+1}|<\varepsilon,\forall n > N$

$|\frac{a_1+a_2+...a_n+n·a_{n+1}}{n(n+1)}|<\varepsilon \Rightarrow |\frac{a_1+a_2+...a_n}{(n+1)n}-\frac{a_{n+1}}{n+1}|<\varepsilon$

于是有

$\frac{a_1+a_2+...a_n+a_{n+1}}{n(n+1)}-\varepsilon < \frac{a_{n+1}}{n+1} < \frac{a_1+a_2+...a_n+a_{n+1}}{n(n+1)} +\varepsilon$

$由\ntinf \frac{a_1+a_2+...+a_n}{n}=a,故对上述\varepsilon > 0,\exists N_2 \in N$

$a-\varepsilon <\frac{a_1+a_2+...+a_n}{n}<a+\varepsilon, \forall n>N_2 $

故有
$\frac{a-\varepsilon}{n+1}-\varepsilon<\frac{a_{n+1}}{n+1}<\frac{a+\varepsilon}{n+1}+\varepsilon$

因为
$\ntinf \frac{1}{n+1}=0,故\exists N_3 \in N,\frac{a-\varepsilon}{n+1}>-\varepsilon,\frac{a+\varepsilon}{n+1}<\varepsilon$

取$N=max(N_1,N_2,N_3)则有$

$-2\varepsilon<\frac{a_{n+1}}{n+1}<2\varepsilon,\forall n>N$

故$\ntinf \frac{a_n}{n}=0$

$\nl$
Cauchy收敛准则

$\xtx{\infty} f(x)存在 \Leftrightarrow \forall \varepsilon >0,\exists A>0,\forall x'>A,\forall x''>A有|f(x')-f(x'')|<\varepsilon$

$\nl$
方法2:浙大法

$\frac{a_n}{n}=\frac{a_1+a_2+...+a_n}{n}-\frac{a_1+a_2+...+a_{n-1}}{n-1}·\frac{n-1}{n}$

$同取lim,则有\ntinf \frac{a_n}{n}=a-a·1=0$

$\\$

Ex.2.
$(2-x_n)x_{n+1}=1$

$x_1 \ne 2, x_1 \ne 1$

$x_1<1 \to x_2<1 \to x_2>x_1$ (*)

$x_1>2 \to x_2<1 \to 同上*$ (**)

$1<x_1<2 \to x_2>x_1 \to x_2>2 \to 同(**)$

$\ \ \ \ \ \ \ \ \ \qquad \qquad \qquad \to x_2<2 \to x_3<2 \to x_n <2 则 \ntinf x_n \le 2 \to a=\frac{1}{2-a},a=1矛盾$

$\ \ \ \ \qquad \qquad \qquad \qquad \qquad \qquad  \to x_3>2 \to x_n>2 同(**) \ntinf x_n \ge 1 $
 
方法2
$x_1 \ne 1,有\frac{1}{x_{n+1}-1}=\frac{1}{x_n-1}-1=....=\frac{1}{x_1-1}-n$

$推出x_{n+1}-1=\frac{1}{\frac{1}{x_1-1}-n} \to 0 (n \to \infty)$

$\nl$
二、O'stolz定理及其应用
$\nl$
定理1.
$设\{b_n\}是严格单调的递减数列,且\ntinf a_n=\ntinf b_n = 0$

则当$\ntinf \frac{a_n-a_{n+1}}{b_n-b{n+1}}存在(有限或无穷)时,\ntinf \frac{a_n}{b_n}也存在,且$

$\ntinf \frac{a_n}{b_n} = \ntinf \frac{a_n-a_{n+1}}{b_n-b{n+1}} $

$\nl$
定理2.

$设\{b_n\}为严格递增的数列,且\ntinf b_n = + \infty$

则当$\ntinf \frac{a_n-a_{n+1}}{b_n-b{n+1}}存在(有限或无穷)时,\ntinf \frac{a_n}{b_n}也存在,且$

$\ntinf \frac{a_n}{b_n} = \ntinf \frac{a_n-a_{n+1}}{b_n-b{n+1}} $

$\nl$
用法示例

$\nl$
$①证明\ntinf a_n=a \to \ntinf \frac{a_1+a_2+....+a_n}{n}=a$

$令 \frac{x_n}{y_n}=\frac{a_1+a_2+....+a_n}{n}$

$\nl$
$②\ntinf \frac {\sqrt{1}+\sqrt{2}+....+\sqrt{n}}{n \sqrt{n}}  $

$= \ntinf \frac{\sqrt{n+1}}{(n+1)\sqrt{n+1}-n\sqrt{n}}$

$= \ntinf \frac{\sqrt{n+1}[(n+1)\sqrt{n+1}+n\sqrt{n}]}{(n+1)^3-n^3}$

$= \frac{2}{3}$

$\nl$
$③\ntinf \frac{ln n}{n}= \ntinf \frac{ln(n+1)-lnn}{n+1-n}=\ntinf ln(1+\frac{1}{n})=0$

$\nl$
$④\ntinf \frac {n^2}{a^n} (a>1)= \ntinf \frac{(n+1)^2-n^2}{a^{n+1}-a^n} =$

$\ntinf \frac {2n+1}{a^n(a-1)} = \frac{1}{a-1} \ntinf \frac{2}{a^{n+1}-a^n}=0$

$\nl$
$⑤0<x_n<1, x_{n+1}=x_n(1-x_n)$

求证$\ntinf nx_n=1$

证明:

$1^\circ 证明x_n递减有界,0<x_n(1-x_n)<x_n<x_1 极限存在,求得 \ntinf x_n =0$

$2^\circ \{ \frac{1}{x_n} \} 严格递增,极限 \to + \infty$

$3^\circ nx_n = \frac{n}{\frac{1}{x_n}} \overset{\underset{\mathrm{stolz}}{}}{=} \frac{1}{\frac{1}{x_n}-\frac{1}{x_{n-1}}}$

$=\frac{x_nx_{n-1}}{x_{n-1}-x_n}= \frac{x_nx_{n-1}}{x_{n-1}x_{n-1}} = \frac{x_n}{x_{n-1}}=1-x_{n-1} \to 1$


$\nl$
$⑥x_1=sinx,x_{n+1}=sinx_n$

$利用(\xtx{0}\frac{x^2sinx^2}{x^2-sinx^2}=3) \to \ntinf nsin^2x_n=3 \to \ntinf \sqrt{\frac{n}{3}}sinx_n=1$

$\nl$
连续函数:P76
注意3项

$x \to x_0,f(x)有极限,区别:x=x_0处未必有定义$

$|f(x)-A|<\varepsilon$

$\nl$

$f(x)在x=x_0处连续$

$|f(x)-f(x_0)|<\varepsilon$

$\nl$

f(x)在a连续等价于

(1)$\xtx{a}f(x)=f(a)$

(2)$\xtx{a+}f(x)=\xtx{a-}f(x)=f(a)$

$\nl$

定义2、设f(x)在$x_0$的某左(右)领域有定义,且
$\xtx{x_0-}f(x)=f(x_0)(或\xtx{x_0+}f(x)=f(x_0))称f(x)在x_0处左(右连续)$

$\nl$
定理1

f(x)在点$x_0$处连续充要条件是:f(x)在点$x_0$处左连续又右连续

符号表示为

$(a,b)内,f(x) \in C(a,b)$

$[a,b]内,f(x) \in C[a,b]$

$\bold R上,f(x) \in C \bold R$

$\nl$
Ex
$\nl$(1)
$\xtx{0-}f(x)=+\infty$

$\xtx{0+}f(x)=0$

故f在$x_0$整体不连续

一致函数反函数一般不再一致连续

$\nl$
(2)
$y=\frac{1}{x}-[\frac{1}{x}]$

$\nl$
(3)
$
f(x)=\begin{cases}
sin\pi x, & x\in Q \\
0, & x\notin Q
\end{cases}
$

$f(x)在\bm Z连续,在非\bm Z不连续$

$f(x)在x_0连续 \ne f(x)在x_0附近连续$

$\nl$
二、间断点的分类,P83

$f(x)=\frac{cos\frac{\pi}{2}x}{x^2(x-1)},x=0为二类点$

$x=1,f(x)=-\frac{\pi}{2}$

单调函数只可能有第一类间断点

$\nl$
定理1.

$设u=g(x)在x_0处连续,u_0=g(x_0),又有y=f(u)在u_0处连续,则复合函数f(g(x))在x_0处连续$

$证明:由y=f(u_0)在u_0处连续,\forall \varepsilon >0,\exists \eta >0$

$|f(u)-f(u_0)|<\varepsilon, \forall u(|u-u_0|<\eta)$

$又u_0=g(x_0)及u=g(x)在x_0处连续$

$对上述\eta>0,\exists \delta>0$

$|u-u_0|=|g(x)-g(x_0)|<\eta,\forall x(|x-x_0|<\varepsilon)$

此时有
$|f(g(x))-f(g(x_0))|<\varepsilon,\forall x(|x-x_0|<\varepsilon)$

$\nl$
四、函数一致连续

P86(必须在区间上,公用的$\delta$)

$f(x)=x,x \in R, \forall x_0 \in R, \forall \varepsilon>0, \exists \delta=\varepsilon$

$|f(x)-f(x_0)|=|x-x_0|<\varepsilon,\forall x(|x-x_0|<\delta)$


$f(x)=x^2,x \in R, \forall x_0 \in R$

$|f(x)-f(x_0)|=|x-x_0|·|x+x_0|见P78$

数列型反例
$x_1=n,x_2=n+\frac{1}{n}$

$\nl$
一致连续等价叙述

$\{x_n\}\{y_n\} \in I, \ntinf (x_n)-(y_n)=0,有\ntinf f(x_n)-f(y_n)=0$

sinx与x在R上一致连续,但sinx·x在R上非一致连续

反例
$\forall \delta >0,x_1=2n\pi+\frac{\delta}{3},x_2=2n\pi-\frac{\delta}{3}$

$\nl$

$注:f(x)在I上不一致连续,是指\exists \varepsilon_0 >0, \forall \delta >0$

$|f(x_1)-f(x_2)| \ge \varepsilon_0,当x_1,x_2 \in I,且|x_1-x_2|<\delta$

或即
$\exists \varepsilon_0 >0和 |x_n^{(1)}|,|x_n^{(2)}| \subset I,(\ntinf x_n^{(1)}-x_n^{(2)}=0)$

$|f(x_n^{(1)})-f(x_n^{(2)})| \ge \varepsilon_0$

先找到不一致连续之处,很陡时,$x_n \to 它$

$\nl$
Ex:
$(1)f(x)=sin\frac{1}{x},x \in(0,1)$

$x_n^{(1)}=\frac{1}{2n\pi},x_n^{(2)}=\frac{1}{2n\pi+\frac{\pi}{2}}$

$(2)f(x)=lnx, x \in (0,+\infty)$

$x_n^{(1)}=e^{-n},x_n^{(2)}=e^{-n-1}$

$\\$

$\nl$
五、闭区间上连续函数性质

定理3(最值定理P85)推论:有界

$\nl$
例6(P90)设
$f(x) \in C[a,+\infty),\xtx{+\infty}f(x)存在,证明f(x)在[a,+\infty)上有界$

$证明:设\xtx{\infty}f(x)=l,令\varepsilon_0=1,必存在A>a,当x>A时$

$|f(x)-l|<1 \to |f(x)<1+|l|$

$又f(x) \in C[a,A],从而有界,记|f(x)|\le M,\forall x \in [a,A]$

$|f(x)| \le max(M,1+|l|)$

$\nl$
定理4(零点存在,P85)

$x^3+4x^2-3x-1=0$

$f(0)=-1,f(1)=1,f(-1)=5$

$\xtx{-\infty}f(x) \to -\infty,由局部有界性可知,存在充分大大A>0$

$使f(-A)<0,故有第3根$

$\nl$
介值定理

$Ex:f(x) \in C[a,b],且x_1,x_2,..x_n \in [a,b]$

$存在\zeta \in [a,b],有f(\zeta)=\frac{f(x_1)+f(x_2)+...f(x_n)}{n} \triangleq A$

$利用min(f(x)) \le A \le max(f(x))可证$

$\nl$
康托定理P115

在(a,b)内连续的函数f(x)在(a,b)内一致连续的充要条件是

$\xtx{a^{+0}}f(x)和\xtx{b^{-0}}f(x)均存在$

$1^\circ 必要性$

用柯西收敛准则

由定义可知
$\forall \varepsilon >0,\exists \delta >0,(0<\delta<b-a)$

$|f(x')-f(x'')|<\varepsilon,\forall x',x'' \in (a,b),且|x'-x''|<\delta$

$考虑(a,a+\delta)内任意两点x',x'',也即有$

$a<x'<a+\delta$

$a<x''<a+\delta$

此时也有
$|x'-x''|<\delta,从而|f(x')-f(x'')|<\varepsilon$

由cauchy准则可知
$\xtx{a^{+0}}f(x)存在$

$2^\circ 充分性$
作辅助函数
$
F(x)=\begin{cases}
l_1,& x=a \\
f(x), & a<x<b \\
l_2, & x=b
\end{cases}
$

故F(x)在[a,b]连续,即一致连续

此命题用于判断函数是否一致连续

$\nl$
$例10.P117(*)$

证明:
方法1:
$因f(x) \in C[a,M+1],f(x)在[a,M+1]上一致连续$

$对上述\varepsilon > 0,\exists \delta>0(0<\delta<1),当x',x'' \in [a,M+1],且|x'-x''|<\delta 时$

$有|f(x')-f(x'')|<\frac{\varepsilon}{2}$

当$x' \in [a,M],x'' \in [M,+\infty)且|x'-x''|<\delta ,$

$因为|x'-M|\le|x'-x''|<\delta,而x'' \ge M$

应有
$|f(x')-f(x'')|\le|f(x')-f(M)|+|f(x'')-f(M)|<\frac{\varepsilon}{2}+\frac{\varepsilon}{2}=\varepsilon$

$\forall x',x'' \in [M,+\infty)类似$

$\forall x',x'' \in [a,M]类似$

方法2:
因
$\xtx{+\infty}f(x)存在,由柯西准则可知$

$\forall \varepsilon>0,\exists M>a:当x',x'' \ge M时,有|f(x')-f(x'')|<\frac{\varepsilon}{2}$

$f(x) \in C[a,M+1],从而f(x)在[a,M+1]上一致连续$

$对上述\varepsilon > 0,\exists \delta (0<\delta<1)$

使对$\forall x',x'' \in [a,M+1],当|x'-x''|<\delta 时$

$有|f(x')-f(x'')|<\frac{\varepsilon}{2}$

$\\$
$\nl$
习题课
(1)构造函数f(x)

①.在R上处处不连续:D(x)

②.仅在0处连续

$
f(x)=\begin{cases}
x, & 有理数 \\
-x, & 无理数
\end{cases}
$

③.在[0,1]上任一无理点连续,任一有理点不连续:黎曼函数

④.在[-1,1]上不连续,但可取到最大值,最小值,一切中间值

⑤.在[-1,1]上处处不连续,但可取到最大值,最小值,一切中间值

对例②函数改造如下

\begin{tikzpicture}
\draw[->] (-1.2,0) -- (1.2,0);
\draw[->] (0,-1.2) -- (0,1.2);
\draw[style=dashed] (-1,-1) -- (1,1);
\draw[style=dashed] (-1,0) -- (0,1);
\draw[style=dashed] (0,-1) -- (1,0);
\node [right] at (1.2,0) {x};
\node [above] at (0,1.2) {y};
\end{tikzpicture}

$\nl$
$设f(x)在u_0处连续,u=g(x)(u+0=g(x_0)在x \to x_0时$

$极限存在,是否必有f(g(x))在x=x_0处连续?$

考虑
$
f(u)=\begin{cases}
usin\frac{1}{u}, & u \ne 0 \\
0, & u=0
\end{cases}
~~ g(x)=\begin{cases}
1, & x \ne 0 \\
0, & x=0
\end{cases}
$

$在x_0=0处不连续$

$\nl$
$例,设f(x)在R上满足Lipschitz条件$

$\forall x,y \in R,有|f(x)-f(y)| \le L·|x-y| (0<L<1)$

$证明f(x)在\bm R上有唯一不动点,也即有$
$f(x_0)=x_0$

$证:构造数列x_{n+1}=f(x_n)$

$|x_{n+1}-x_n| \le L^{n-2}|x_2-x_1|$

$用Cauchy可证x_n收敛,记\xtx{\infty}x_n=x_0$

$x_0=\ntinf x_{n+1}= \ntinf f(x_n)=f(\ntinf x_n)=f(x_0)$

唯一性

$|x_0'-x_0|=|f(x_0')-f(x_0)| \le L·|x_0'-x_0|,矛盾$ 

$\nl$
例:设
$f(x) \in C(a,b)且 \xtx{a^+}f(x)=\xtx{b^-}f(x)= -\infty$

证明:f(x)在(a,b)内必可取到最大值

$取[a+\delta,b-\delta]内最值f(c),\exists \delta>0 (a+\delta<\frac{a+b}{2})$

$f(x)<f(\frac{a+b}{2}),\forall x (a<x<a+\delta _1)$

$\nl$
例:
$f(x) \in C(R)为周期T的周期函数,证明f(x)在R上一致连续(非定理)$

$因f(x)\in C(R),故f(x)在[-T,T]上一致连续,\forall \varepsilon >0,$

$\exists \delta >0,(0<\delta <T) |f(y')-f(y'')|<\varepsilon,\forall y',y'' \in [-T,T]$

$|y'-y''|<\delta,\forall x',x'' \in R,|x'-x''|<\delta$

由f的周期性可知
$\exists y',y'' \in [-T,T]以及n \in Z$

$使得x'=nT+y',x''=nT+y'',且|y'-y''|<\delta$

$于是|f(x')-f(x'')|=|f(nT+y')-f(nT+y'')|=|f(y')-f(y'')|<\varepsilon$

$f(x)在R上不一致连续 \to f(x)不是周期函数$

$\nl$
$例:(sinx)^3+sin(x^3)$

$x_n^1=\sqrt[3]{2n\pi+\frac{\pi}{2}},x_n^2=\sqrt[3]{2n\pi-\frac{\pi}{2}}$

$\newline$

$f(x)满足Lipschitz条件,则\frac{f(x)}{x}一致连续$

$证明:|\frac{f(x)}{x}-\frac{f(y)}{y}| = |\frac{f(x)}{x}-\frac{f(y)}{x}+\frac{f(y)}{x}-\frac{f(y)}{y}| \le \frac{1}{x}|f(x)-f(y)|+|f(y)|·|\frac{1}{x}-\frac{1}{y}|$

$\le \frac{|f(x)-f(y)|}{a}+\frac{|x-y|}{xy}(|f(a)|+k|y-a|)$

$\le \frac{1}{a}(2k+|f(a)|)·|x-y|$

$\newline$

$例:f(x) \in  C(0,+\infty), \forall x \ge 0,有\ntinf f(x+n)=0,证明\ntinf f(x)=0$

$证:由一致连续条件,\forall \varepsilon >0,\exists \delta >0: |f(x)-f(y)|<\frac{\varepsilon}{2}$

$\forall x,y \in (0,+\infty), |x-y|<\delta$

\begin{tikzpicture}
\draw[->] (-1,0) -- (8,0);
\draw (0,0) -- (0,0.5);
\draw (0.4,0) -- (0.4,0.2);
\draw (0.8,0) -- (0.8,1.5);
\draw (1.2,0) -- (1.2,1.5);

\draw[->] (0.4,1.2) -- (0.8,1.2);
\draw[->] (1.6,1.2) -- (1.2,1.2);


\draw (2.8,0) arc (0:180:1);
\draw [dotted] (4.8,0) arc (0:180:1);
\draw (3.2,0) arc (0:180:1);
\draw [dotted] (5.2,0) arc (0:180:1);


\draw (2,0) -- (2,0.5);
\node [below] at (0,0) {0};
\node [below] at (0.4,0) {$x_1$};
\node [below] at (0.8,0) {$x_2$};
\node [below] at (1.2,0) {$x_3$};
\node [below] at (2,0) {1};
\node [above] at (1,1.5) {$<\delta$};
\end{tikzpicture}

$由|f(x)| \le |f(x)-f(x_k+n)|+|f(x_k+n)|$

现将[0,1]区间m等分,分点为$0=x_0<x_1<x_2<...x_m=1$

$且\Delta = x_k-x_{k-1}<\delta (k=1~m)$

$现对\forall x \in [0,1], \exists x_k 使得 |x-x_k| \le \delta $

$对\forall x \in (1,+\infty),\exists n \in N,及其x_k \in [0,1]$

$使|x-(x_k+n)| \le \delta$

$由条件\ntinf f(x+n)=0,对上述\varepsilon >0,\exists N \in \bm N$

$|f(x_k+n)|<\frac{\varepsilon}{2}, n > N$

于是$当x>N+1时,\exists n > N 及其某一x_k$

$使得|x-(x_k+n)| \le \delta$

$|f(x)| \le \frac{\varepsilon}{2}+\frac{\varepsilon}{2}= \varepsilon$

$\nl$

$\S 3.1实数基本定理$

$一、致密性定理,P103$

$二、区间套定理,P101,套一个数出来$

$三、有限覆盖,P106$

$\nl$
$例:设f(x)在(a,b)上定义,对\forall x' \in (a,b) \xtx{x'}f(x)存在,证明f(x)在(a,b)上有界$

$证明:对\forall x' \in (a,b), \xtx{x'}f(x)=A',对\epsilon'=1$

$\exists \delta'=\delta(x',\epsilon')>0,|f(x)-A'|<1,\forall x \in U_0(x',\delta')$

$|f(x)| \le |A'|+1,\forall x \in U(x',\delta'),记M'=max(|A'|+1,f(x'))$

$对U(x',\delta')有|f(x)| \le M',利用有限覆盖定理$

$对开区间集{U(x',\delta'),x' \in (a,b)}...有限集G^*,P107之方法$

$\\$
$①确界 \to ②单调有界 \to ③区间套(柯西-康托) \to ④有限覆盖 $

$\\$
$②单调有界 \to ⑤致密性 \to ⑥柯西收敛 $

$\\$
$① \to ③.记sup{a_n}=\zeta,故a_n \le \zeta,a_n \le b_m,\forall n,m$

$因上确界为最小上界,故\zeta \le b_m,\forall m,n \forall \epsilon>0, \exists N, G_n>\zeta-\epsilon$

$由a_n递增性可知a_n>\zeta-\epsilon,当n>N后,0 \le \zeta-a_n < \epsilon,\forall n>N$

$故|a_n-\zeta|<\epsilon,\forall n >N,即\ntinf a_n = \zeta,同理证出 \ntinf b_n=\zeta'$

$再证\zeta = \zeta'$

$\nl$
$例:③ \to ②.证明:设数列上界b_1,取a_1=x_1,则{x_n} \subset  [a_1,b_1]$

$将[a_1,b_1]平分,至少有一个区间有无穷多项,记为[a_2,b_2]....$

$故由区间套条件满足且[a_n,b_n]含有无穷多项x_n$

$故有lima_n=limb_n=\zeta,因lim(a_n-b_n)=0$

$\forall \epsilon >0,\exists N_0 \in N:|b_{N_0}-a_N|<\epsilon,且[a_{N_0},b_{N_0}]含无穷多项$

$故当n充分大时,以后所有项均在a_{N_0},b_{N_0}内,若不然,则仅有有限在区间内$

$则|x_n-\epsilon| \le |b_{N_0}-a_{N_0}| < \epsilon$

$\nl$

$\S 3.3.上极限,下极限$

$定义,设\{x_n\}有界,记A \le x_n \le B$

$对每一个给定的n \in N,考虑非空数集\{x_k\}_{k \ge n}$

$记\alpha_n=\inf_{k \ge n}\{x_k\},\beta_n=\sup_{k \ge n}\{x_k\}$

$故有(i)A \le \alpha_n \le \beta_m \le B, \forall n,m$

$(ii)\alpha_n \le \alpha_{n+1} \le \beta_{n+1} \le \beta_n$

$\ntinf \alpha_n为下极限,记为\varliminf\limits_{n \to \infty}x_n. \ntinf \beta_n为上极限,记为\varlimsup\limits_{n \to \infty}x_n$

$1.有界数列必有上下极限且唯一$

$2.若数列无上界,则规定上极限为+\infty,若数列无下界,则规定上极限为-\infty$

$\nl$
基本性质$(设x_n,y_n有界)$

$①\varliminf\limits_{n \to \infty}x_n \le \varlimsup\limits_{n \to \infty}x_n$

$②如果x_n \le y_n,则\varliminf\limits_{n \to \infty}x_n \le \varliminf\limits_{n \to \infty}y_n$

$\varlimsup\limits_{n \to \infty}x_n \le \varlimsup\limits_{n \to \infty}y_n$

$③\varliminf\limits_{n \to \infty}x_n + \varliminf\limits_{n \to \infty}y_n \le \varliminf\limits_{n \to \infty}(x_n+y_n) \le \varlimsup\limits_{n \to \infty}(x_n+y_n) \le \varlimsup\limits_{n \to \infty}x_n + \varlimsup\limits_{n \to \infty}y_n$

$④若x_n,y_n>0,\forall n \in N,则\varliminf\limits_{n \to \infty}x_n \varliminf\limits_{n \to \infty}y_n \le \varliminf\limits_{n \to \infty}(x_n y_n) \le \varlimsup\limits_{n \to \infty}(x_n y_n) \le \varlimsup\limits_{n \to \infty}x_n \varlimsup\limits_{n \to \infty}y_n$

$(利用确界不等式证明以上)$

$\nl$
$定理1.等价命题,除以上定义外,另有(仅对于上极限,下极限类似)$

$①\exists \{x_{n_k}\} \subset \{x_n\}为收敛之列,且\ktx{\infty}x_{n_k}'=\beta$

$对\forall {x_{n_k'}}若{x_{n_k'}}收敛,则\ktx{\infty}x_{n_k'}=\beta' \le \beta$

$②\forall \epsilon > 0,\{x_n\}中大于\beta + \epsilon 的项至多有限个$

$大于\beta - \epsilon 的项必有无穷多个。即:\forall \epsilon >0,(\beta - \epsilon,\beta+\epsilon)$

$包含\infty 个x_n,但开区间右边(左边)至多有有限个x_n$

$定理2. \ntinf a_n=a \Leftrightarrow \varlimsup\limits_{n \to \infty}a_n=\varliminf\limits_{n \to \infty}a_n=a$

$\nl$

例:Cauchy准则的上下极限处理(充分性)

$\forall \epsilon > 0, \exists N:x_{N+1} - \epsilon < x_n < x_{N+1} + \epsilon ,\forall n>N$

$令n \to \infty,对\{x_n\}取上下极限$

$\exists N:x_{N+1} - \epsilon \le \varliminf\limits_{n \to \infty}x_n \le \varlimsup\limits_{n \to \infty}x_n \le x_{N+1} + \epsilon$

$故 0 < \beta - \alpha \le 2\epsilon$

$故\beta = \alpha$

$\nl$

$例:设x_n>0,证明:\varlimsup\limits_{n \to \infty}n(\frac{1+x_{n+1}}{x_n}-1)  \ge 1$

$反证法:若\varlimsup\limits_{n \to \infty}n(\frac{1+x_{n+1}}{x_n}-1)  < 1$

$\exists N \in \mathbf{N}使得当n>N时有 n(\frac{1+x_{n+1}}{x_n}-1)<1$

$即\frac{x_n}{n} > \frac{1}{n+1}+\frac{x_{n+1}}{n+1},(n \ge N)$

$故\frac{x_n}{n} > \frac{1}{n+1}+\frac{x_{N+1}}{N+1} > \frac{x_{N+2}}{N+2}+\frac{1}{n+1}+\frac{1}{n+2}$

$>\frac{1}{N+1}+\frac{1}{N+2}+\frac{1}{N+3}+...+\frac{1}{N+P}+\frac{x_{N+P}}{N+P}$

$\frac{x_N}{N} > \frac{1}{N+1}+\frac{1}{N+2}+\frac{1}{N+3}+...+\frac{1}{N+P}+\frac{x_{N+P}}{N+P}$

$令P \to \infty 则常数 > +\infty 矛盾$

$\nl$

$例:设\{x_n\}满足0 \le x_{m+n} \le x_m+x_n,\forall m,n \in N$

$证明\{\frac{x_n}{n}\}收敛$

$0 \le x_n \le x_{n-1}+x_1 \le ... \le nx_1,即0 \le \frac{x_n}{n} \le x_1$

$故\{\frac{x_n}{n}\}有界$

$因为有界数列必有上下极限,故设$

$\varlimsup\limits_{n \to \infty}\frac{x_n}{n}=\beta,\varliminf\limits_{n \to \infty}\frac{x_n}{n}=\alpha$

$只需证\beta \le \alpha 即可$

$任意固定m,对\forall n(充分大)有n=q_nm+r_n(0 \le r_n < n),其中q_n,r_n 均为\mathbf{Z}$

$故x_n=x_{q_nm+r_n} \le q_n x_m + x_{r_n}$

$\frac{x_n}{n} \le \frac{q_n x_m}{n} + \frac{x_{r_n}}{n}=\frac{q_n x_m + x_{r_n}}{q_nm+r_n}$

$\le \frac{x_{r_n}}{q_n x_m}+\frac{x_{r_n}}{q_n m} = \frac{x_m}{m}+\frac{x_{r_n}}{q_n m}$

$故\frac{x_n}{n} \le \frac{x_m}{m}+\frac{x_{r_n}}{q_n m} \le \frac{x_m}{m}+\frac{x_1{r_n}}{q_n m}$

$\le \frac{x_m}{m}+\frac{x_1}{q_n}$

$令n \to \infty 则q_n也 \to \infty,上式两边同取上极限,有$

$\beta = \varlimsup\limits_{n \to \infty}\frac{x_n}{n} \le \varlimsup\limits_{n \to \infty}(\frac{x_m}{m}+\frac{x_1}{q_n}) \le \varlimsup\limits_{n \to \infty}\frac{x_m}{m}+\varlimsup\limits_{n \to \infty}\frac{x_1}{q_n}=\frac{x_m}{m}+0$ 

$再m \to \infty,取下极限 \beta \le \varliminf\limits_{n \to \infty}\frac{x_m}{m} = \alpha$

$\nl$

$实数基本定理$

$一.确界原理基本应用$

$例:设f(x)在[a,b]上单调递增,f(a) \ge a,f(b) \le b,证明:f(x)在[a,b]上有不动点(不连续)$

$记E=\{x|f(x) \ge x,x \in [a,b]\},故E非空且有上界$

$记x_0= supE,\forall x \in 有x \le x_0,单调递增$

$故f(x) \le f(x_0),故x \le f(x) \le f(x_0)$

$故f(x_0)为E上界,x_0 \le f(x_0)(上确界不大于任一上界)$

$因f(x)递增,a \le x_0 \le f(x_0) \le f(b) \le b$

$故f(x_0) \le f(f(x_0))$

$故f(x_0) \in E,f(x_0) \le x_0,即f(x_0)=x_0$

$\nl$

$例:① \to ④$

$记E=\{x|[a,x]为G有限覆盖,a < x \le b\}$

$设a \in (\alpha,\beta) \in G,故\exists x > a:(a,x) \subset (\alpha, \beta)$

$记\beta = supE \le b$

$1^\circ 可证\beta \in E,2^\circ \beta =b$

$\nl$

$二、致密性定理$

$⑤ \to ②$
$证明:\exists \{x_{n_k}\} \subset \{x_n\}, \ktx{\infty} x_{n_k}=a$

$\forall \epsilon >0,\exists K,a-\epsilon < x_{n_k} < a+\epsilon, \forall k \ge K$

$\forall n > n_k \ge K ,x_n \ge x_{n_k} > a-\epsilon$

$由\ktx{+\infty}n_k=+\infty,故\exists x_{n_{k'}}>n$

$a-\epsilon < x_{n_k} \le x_n \le x_{n_{k'}} < a+\epsilon$

$\nl$

$例:f(x)在[a,b]上只有第一类间断点,证明其有界$

$设无界,\forall M_n, \exists |f(x_n)|>M_n,取M_n=n$

$因\{x_n\}有界,故\{x_n\}有收敛子列,\ktx{\infty}x_{n_k}=x_0 \in [a,b]$

$\ktx{\infty}f(x_{n_k})=\infty,故\xtx{x_0}f(x)不存在$

$\nl$

$例:设f(x) \in C[a,b]且有唯一最值点,x_0 \in [a,b],若有\{x_n\} \subset [a,b]$

$使得\ntinf f(x_n)=f(x_0),证明\ntinf x_n = x_0$

$反证法,若\ntinf x_n \ne x_0,则\exists \epsilon_0 >0,以及\{x_{n_k}\} \subset (a,b)$

$|x_{n_k}-x_0| \ge \epsilon_0,\forall n \in N,现\{x_{n_k}\}有界,故存在收敛子列$

$不妨记其为自身,记\ntinf x_{n_k}=x_1 \ne x_0$

$f(x_1)=\ntinf f(x_{n_k})=\ntinf f(x_n)=x_0$

$\nl$

$例:一致连续 \Leftrightarrow \{x_n\}Cauchy,\{f(x_n)\}也Cauchy$

$\nl$

$三、有限覆盖定理之应用$

$④ \to ⑤$

$设E为直线上点集,\forall \epsilon >0,U_0(x_0,\epsilon)中含有G中无限个点,称为聚点$

$\nl$

$④ \to ③$

$证明:对\forall  x' \in [a,b],至少存在[a_k,b_k], x' \notin [a_k,b_k]$

$即对x',\exists U(x')与[a_k,b_k]交集为空$

$G=\{U(x',\delta_{x'})\}设G中有限个U(x)覆盖住[a,b]$

$故U(x_i,\delta_i) \cap [a_{k_i},b_{k_i}]=\emptyset,i=1~n,k_i=2~n$

$令max(k_1,k_2,...k_n)=k_0$

$则\bigcap_{i=1}^{n}[a_{k_i},b_{k_i}]=[a_{k_0},b_{k_0}]$

$故(x_i,\delta_i) \cap [a_{k_0},b_{k_0}] = \emptyset,i=1~n$

$故[a_{k_0},b_{k_0}] \subseteq [a,b]$

$\nl$

$四、区间套定理$

$③ \to ⑥$

$证明:先证\{x_n\}有界,记为a \le x_n \le b,\forall n \in N$

$用c,d将[a,b]三等分,则左右两侧区间至少有一个含\{x_n\}中有限项,否则$

$\exists N,\exists m,n > N,|x_n-x_m|>|d-c|>\epsilon,与Cauchy列定义矛盾$

$设[a,c]中含有有限项,则取[c,b]为[a_1,b_1],以此类推,故[a_k,b_k]具有性质①)...②)区间长度为\frac{2}{3}$

$③\exists N_k \in N,x_n \in [a_k,b_k],\forall n>N_k$

$故\forall \epsilon >0,\exists K,[a_k,b_k] \subset [\zeta,\epsilon ^K],令③中N=N_k则$

$n>N时,有x_n \in [a_k,b_k]_c \cup U(\zeta,\epsilon) \Rightarrow |x_k-\zeta|<\epsilon,\forall n>N$

$\nl$

$实数基本定理应用$

$例:[a,b]上实数不可排列$

$反证:若可列,记为\{x_n\},[a,b]=\bigcup_{i=1}^{\infty}\{x_i\}$

$将[a,b]三等分,至少有一个区间不含x_1,记为[\alpha_1,\beta_1]$

$将其三等分,不含x_2区间记为[\alpha_2,\beta_2]....则[a_k,b_k]不含有x_1,x_2,...x_k$

$\ktx{\infty}[a_k,b_k]=\zeta,\zeta \in [a,b],\zeta \ne x_n,\forall n$

$\nl$

$例:f(x) \in C[a,b],f(x)在[a,b]上处处极值,则为常数$

$介值定理。若f(x)不恒为常数,则\exists a_1,b_1 \in [a,b],a_1 < b_1$

$不妨设f(a_1)<f(b_1),则\exists [a_2,b_2]:f(a_1)<f(a_2)<f(b_2)<f(b_1)$

$b_2-a_2< \frac{1}{2} (b_1-a_1)$

$f(a_n)严格递增,f(b_n)严格递减$

$\ntinf{a_n} = \ntinf{b_n} =\zeta , 故\zeta 不为局部极值点$

$\nl$

$思考:设f(x)定义于[a,b],\forall x' \in [a,b],\xtx{x'}f(x)存在.证明\forall \epsilon >0$

$[a,b]上能使该点函数值与该点极限值之差的绝对值大于 \epsilon 的点有限$

$|\xtx{x'}f(x)-f(x')|>\epsilon$

$\nl$

$半连续函数$

$设f(x)在[a,b]上定义,\forall \epsilon >0,\exists \delta >0$

$f(x)<f(x_0)+\epsilon,\forall x(|x-x_0|<\delta)$

$称之为在x_0上半连续(f(x)>f(x_0)-\epsilon,下半连续)$

$
例:
f(x) = 
\begin{cases} 1, & 0 \le x \le 1 \\ 0, & -1 \le x < 0 
\end{cases}
为上半连续
$

$\nl$

$例:f(x)在[a,b]上位上半连续,则有$

$(1)有上界(2)有最大值(3)未必有下界。证明,用\epsilon -N语言$

$反例
f(x) = 
\begin{cases} -\frac{1}{1-x}, & x \ne 1 \\ 0, & x = 1 
\end{cases}
$

$\nl$

$问题:①设f(x)在[a,b]上半连续,则f(x)必在[a,b]内某一子区间上有界(区间套)$

$②设f(x)在[a,b]上班连续,g(x)下半连续,f(x) \le g(x)$

$证明\forall \epsilon >0,\exists \delta >0: \forall x',x'' \in [a,b],|x'-x''|< \delta$

$f(x')<g(x'')+\epsilon$


\end{document}