\documentclass[12pt,a4paper]{article}
\usepackage{fontspec}
\usepackage{amsmath}
\usepackage{amssymb}
\usepackage{bm}
\usepackage{tikz}
\setmainfont{Adobe Kaiti Std}
\thispagestyle{empty}
\pagestyle{empty}
\begin{document}

\fancyfoot[C]{by chinasjtu@msn.com }

\newcommand{\nl}{\newline}

\newcommand{\ntinf}{\lim\limits_{n \to \infty}}
\newcommand{\xtinf}{\lim\limits_{x \to \infty}}

\newcommand{\Atinf}{\lim\limits_{A \to \infty}}
\newcommand{\Rtinf}{\lim\limits_{R \to \infty}}

\newcommand{\ntx}[1]{\lim\limits_{n \to #1}}
\newcommand{\xtx}[1]{\lim\limits_{x \to #1}}
\newcommand{\ttx}[1]{\lim\limits_{t \to #1}} 
\newcommand{\ktx}[1]{\lim\limits_{k \to #1}} 
\newcommand{\dxtx}[1]{\lim\limits_{\Delta x \to #1}}

\newcommand{\jfab}{\int_{a}^{b}}
\newcommand{\jf}[2]{\int_{#1}^{#2}}

\newcommand{\nsum}[2]{\sum\limits_{n=#1}^{#2}}
\newcommand{\isum}[2]{\sum\limits_{i=#1}^{#2}}
\newcommand{\ksum}[2]{\sum\limits_{k=#1}^{#2}}

\newcommand{\nsuminf} {\nsum{1}{\infty}}
\newcommand{\ksuminf} {\ksum{1}{\infty}}
\newcommand{\isuminf} {\isum{1}{\infty}}





习题课

两个重要极限证法?

$\frac{sinx}{x},(1+x)^{\frac{1}{x}},用Hospital法则?$

$因为(sinx)'=cosx,利用了\xtx{0}\frac{sinx}{x}=1,导致循环论证$

$f(x)在[a,+\infty)可导,\xtinf f(x)存在,则必有\xtinf f'(x)=0$

导数有界,则一致连续

$设f(x)在(0,a]上可导,且\xtx{0+}\sqrt{x} f'(x)存在,证明f(x)在(0,a]上一致连续$

$\nl$

$f(xy)=f(x)+f(y),f'(1)存在,判断f'(x)存在性$

因为$f(1)=2f(1),故f(1)=0$

$设f'(x)=\frac{f(x+\Delta x)-f(x)}{\Delta x}_{\Delta x \to 0}=\frac{1}{x}\frac{f(1+\frac{\Delta x}{x})-f(1)}{\frac{\Delta x}{x}}_{\Delta x \to 0}=\frac{f'(1)}{x}_{\Delta x \to 0}$

$故f'(x)=\frac{f'(1)}{x}$

$\nl$

$y=sin(A·arcsinx),y'=cos(A·arcsinx)A·\frac{1}{\sqrt{1-x^2}}$

$故(1-x^2)y''-xy'+A^2y=0,故y^{(2k)}(0)=0$

\[ y^{(2k+1)}=(-1)^kA \prod_{i=1}^k (A^2-(2i-1)^2) \]

$\nl$

$f''(x)>0,f(0)<0,求证:\frac{f(x)}{x}在(0,+\infty)递增$

$设x_2 > x_1 > 0,则\frac{f(x_2)}{x_2}-\frac{f(x_1)}{x_1}=[\frac{f(\zeta)}{\zeta}]'(x_2-x_1)$

$=\frac{f'(\zeta)\zeta-f(\zeta)}{\zeta^2}(x_2-x_1)$

$因为x_2-x_1>0,\zeta^2>0$

$又因为f(x)<f(x)-f(0)=f'(\zeta)x<f'(x)x$

$故f'(\zeta)\zeta - f(\zeta) > 0$

$\nl$

证明不等式的方法

利用导数定义,中值定理

Taloy定理,单调性

极值最值,凸函数

$f(x) \in D[0,1],f(0)=f(1)=0,minf(x)=-1,求证:maxf''(x) \ge 8$

$设f(x_0)=-1,则f'(x_0)=0$

$则f(0)=f(x_0)+\frac{f''(\zeta_1)}{2}x_0^2, \to 2=f''(\zeta_1)x_0^2$

$f(1)=f(x_0)+\frac{f''(\zeta_1)}{2}(1-x_0)^2, \to 2=f''(\zeta_2)(1-x_0)^2$

$显然f''(\zeta_1)与f''(\zeta_2) \ge 0$

$若f''(\zeta_1) \ge f''(\zeta_2),则x_0^2 \le (1-x_0)^2,则x_0 \le \frac{1}{2}$

$故有f''(\zeta_1) \ge 8$ 

$\nl$
$设f(x+t)=f(x)+f'(x+\theta t)·t (0<\theta<1)$

$f''(x)连续恒不为0,证明\ttx{0} \theta = \frac{1}{2}$

因为$f(x+t)=f(x)+f'(x)t+\frac{1}{2}f''(x+\theta^*t)t^2$

又因为$f(x+t)=f(x)+f'(x+\theta t)·t$

相减消t,$\ttx{0}\frac{f'(\theta t +x)-f'(x)}{\theta t}\theta=\frac{1}{2}f''(x)$

$\nl$

$设t>0,f(x)有n+2阶连续导数,且f^{(n+2)}(x) \ne 0,\forall x$

$设f(x+t)=f(x)+f'(x)t+\frac{f^{(n+1)}(x)}{n!}t^n+\frac{f^{(n+1)}(x+\theta t)}{(n+1)!}t^{n+1},0<\theta<1$

$\nl$

$例:\xtinf f(x)=A,\xtinf f'''(x)=0,证明:\xtinf f'(x)=\xtinf f''(x)=0$

$f(x+1)=f(x)+f'(x)+\frac{1}{2}f''(x)+\frac{1}{6}f'''(\zeta_1),x<\zeta_1<x+1$

$故f'(x)+\frac{1}{2}f''(x)=0$

$f(x-1)=f(x)-f'(x)+\frac{1}{2}f''(x)-\frac{1}{6}f'''(\zeta_2),x-1<\zeta_2<x$

$故f'(x)-\frac{1}{2}f''(x)=0$

$\nl$

$设f(x)在=0某领域内二阶可导,且\xtx{0}(\frac{sin3x}{x^3}+\frac{f(x)}{x^2})=0$

$求f'(0),f''(0)$

$\frac{sin3x}{x^3}+\frac{f(x)}{x^2}=o(1) \to sin3x+xf(x)=o(x^3),(x \to 0)$

$f(x)=-3+\frac{27}{3!}x^2+o(x^2),(x \to 0)$

$\xtx{0}[\frac{3}{x^2}+\frac{f(x)}{x^2}]=\xtx{0}[\frac{3}{x^2}+\frac{-3+\frac{9}{2}x^2+o(x^2)}{x^2}]$

$\nl$

思考题

$a^{b^a}>b^{a^b},(b>a>1)$

$令x=\frac{lna}{lnb},x>1,y=lnb>0$

$则原式等价于lnx>y(xe^y-e^{xy}),令f(y)=xe^y-e^{xy}$

$f'(y)=x(e^y-e^{xy})<0,故f(y)<f(0)=x-1,故f(y) \le 0时,有lnx>yf(y)$

$\nl$

$例:设f(x)在[a,+\infty)上二阶可导,记M_k=sup_{x \in [a,+\infty)}|f^{(k)}(x)| < +\infty,k=0,1,2,...$

$证明:M_1^2 \le 4 M_0 M_2$

$对\forall x \in [a,+\infty),\forall t>0$

$f(x+2t)=f(x)+f'(x)·2t+\frac{f''(\zeta)}{2}·4t^2,x<\zeta<x+2t$

$故|f'(t)| \le tM_2+\frac{M_0}{t},\forall x \in [a,+\infty),\forall t>0$

$取t=\sqrt{\frac{M_0}{M_2}},|f'(x)| \le 2\sqrt{M_0M_2},\forall x \in (a,+\infty)$

$\nl$

指数型辅助函数构造

$(1)f(x)在R上可导,f'(x)+f(x)>0$

$证明f(x)在R上至多只有一个根$

$反证法,若有两个根x,x',令F(x)=f(x)e^x,F'(x)>0,故F严格递增,故不可能有两个根$

$(2)f(x)在R上可导,则\forall k \in R,在f(x)=0的两根之间必有f'(x)-kf(x)=0的一个根$

$构造F(x)=f(x)e^{-kx}$

$设f(x)两根为x_1,x_2,故F(x_1)=F(x_2)=0$

$\exists F'(x)=f'(x)e^{-kx}+f(x)e^{-kx}(-k)=e^{-kx}(f'(x)-kf(x))=0$

$\nl$

$设f(x) \in (a,b),在(a,b)内可导,f(a)f(b)>0,f(a)f(\frac{a+b}{2})<0$

$则对\forall k \in R,\exists \zeta \in (a,b)使得\frac{f'(\zeta)}{f(\zeta)}=k,f(x) \ne 0$

$解法同上,注:在a与\frac{(a+b)}{2}与b之间存在两根$

$\nl$

$例:f(x)在[a,+\infty)上可导,且\xtinf[f(x)+f'(x]=0$

$证明:\xtinf f(x)=0$

$分析:\forall \epsilon >0,\exists A>0,当x>A,|f(x)+f'(x)|<\epsilon$

$利用Cauchy中值定理,构造\frac{F(x)}{G(x)} \to \frac{f(x)·e^x}{e^x}$

$\frac{F(x)-F(A)}{G(x)-G(A)}=\frac{F'(\zeta)}{G'(\zeta)} \to \frac{f(x)e^x-f(A)e^A}{e^x-e^A}=f'(\zeta)+f(\zeta)$

$|f(x)| \le e^{A-x}|f(A)|+|f'(\zeta)+f(\zeta)|(1+e^{A-x})$

$\xtinf e^{A-x}=0,\exists A_1 > A,e^{A-x}<\epsilon,e^{A-x}<1,\forall x>A$

$|f(x)| \le (|f(A)|+2)\epsilon$

$\nl$

$设f(x)在x_0处有2m+1阶导数,且$

$f'(x_0)=f''(x_0)=...f^{(2m)}(x_0)=0,f^{(2m+1)}(x_0)>0$

$证明f(x)在x_0处递增,即\exists \delta>0,f(x)<f(x_0),\forall x\in (x_0-\delta,x_0)$

$f(x)>f(x_0),\forall x\in (x_0,x_0+\delta)$

$因为f(x)=f(x_0)+f'(x_0)(x-x_0)+....$

$f(x)-f(x_0)=\frac{f^{(2m+1)}(x_0)}{(2m+1)!}(x-x_0)^{2m+1}+o(x-x_0)^{2m+1}$

$=(x-x_0)^{2m+1}(\frac{f^{(2m+1)}(x_0)}{(2m+1)!}+o(1))$

$因为\frac{f^{(2m+1)}(x_0)}{(2m+1)!}+o(1)>0,故x \to x_0时,[f(x)-f(x_0)]与x-x_0同号$

$\nl$

$设f(x)在[a,+\infty)上二阶可导,f(a)>0,f'(a)<0,f''(x) \le 0,\forall x>a$

$证明:f(x)在(a,+\infty)上有且仅有一个实根$

$当x>a,f'(x)<0 \to 当x>a,f(x)递减$

$f(x)-f(a) = f'(\zeta)(x-a) \le 0$

$f(x)=f(a)+f'(\zeta)(x-a) \le f(a)+f'(a)(x-a)$

$因为f'(\zeta)<f'(a)<0$

$当x \to +\infty, 由f'(a)<0 \to f(x) \to -\infty,故恰有一根$

$\nl$

$设f(x),g(x)在[a,b]上可导$

$f(x)g'(x)-f'(x)g(x) \ne 0.\forall x \in [a,b]$

$证明:f(x)两零点之间至少有g(x)一个零点$

$证:若g(x)无零点,令F(x)=\frac{f(x)}{g(x)},求得F'(x)$

$设f(x_1)=f(x_2)=0,且此时g(x_1) \ne 0,g(x_2) \ne 0$

$则F(x_1)=F(x_2)=0,用Rolle定理$

$\nl$

对称函数的常数k值

$用途:证明 \exists \zeta \in (a,b),f^{(n)}(\zeta)=k,n=2,3....$

$条件:函数表达式具有对称性$

$方法:1.将端点a(或b)改为x,相应函数值也改为f(x)$

$2.将改变后的函数记为F(x)$

$3.F(x)满足中值定理条件$

$例:f(x) \in [a,b],(a,b)可导,(0<a<b)证明:\exists \zeta \in (a,b)$

$使得f(b)-f(a)=(ln \frac{b}{a})\zeta f'(\zeta)$

$旧方法:即\frac{f(b)-f(a)}{lnb-lna}=\zeta f'(\zeta)=\frac{f'(\zeta)}{\frac{1}{\zeta}},用Cauchy$

$新方法:记\frac{f(b)-f(a)}{lnb-lna}=k,故f(x)-lnx·k关于a,b具有对称性,f(a)-lna·k=f(b)-lnb·k$

$作F(x)=f(x)-lnx·k-f(a)+lna·k$

$F(a)=F(b)=0(由k定义),用Rolle定理,F'(\zeta)=0,即f'(x)=\frac{k}{x}$

$\nl$

$例:f(x)在[a,b]三阶可导,证明\exists \zeta \in (a,b),使得$

$f(b)=f(a)+f'(\frac{a+b}{2})(b-a)+\frac{1}{24}f'''(\zeta)(b-a)^3$

$令k=f'''(\zeta)=...该式关于a,b对称$

$令b=x,F(x)=f(x)-f(a)-f'(\frac{b+x}{2})(x-a)-\frac{k}{24}(x-a)^3$

$判断有无F(a)=F(b)=0$

$F'(\zeta)=f'(\zeta)-f'(\frac{a+\zeta}{2})-\frac{f''(\frac{a+\zeta}{2})}{2}(\zeta-a)-\frac{k}{8}(\zeta-a)^2=0$

$将f'(x)在\frac{a+\zeta}{2}处展开$

$f'(x)=f'(\frac{a+\zeta}{2})+f''(\frac{a+\zeta}{2})(x-\frac{a+\zeta}{2})+\frac{1}{2}f'''(\zeta')(\frac{a+\zeta}{2})^2$

$令x=\zeta,再代入上式有k=f'''(\zeta')$

$\nl$

$思考题:1.设f(x)在(a,b)上可导,a<x<b,证明\exists \zeta \in (a,b)使得$

$\frac{1}{b-a} \begin{vmatrix} b^n & a^n \\ f(a) & f(b) \end{vmatrix}=[nf(\zeta)+\zeta f'(\zeta)]\zeta^{n-1}$

$2.设f(x)在闭区间三阶可导,证明$

$f(b)=f(a)+\frac{1}{2}(b-a)[f'(a)+f'(b)]-\frac{f'''(\zeta)}{12}(b-a)^3$

$\nl$

$设f(x)在[0,+\infty)上可导,且0  \le f(x) \le \frac{x}{1+x^2},证明\exists \zeta \in (0,+\infty)$

$设f'(\zeta)=\frac{1-\zeta^2}{(1+\zeta^2)^2} \Leftrightarrow f'(\zeta)-(\frac{\zeta}{1+\zeta^2})'=0$

$F(x)=f(x)-\frac{x}{1+x^2},且F(0)=0,F(+\infty)=0$

$由推广的Rolle定理,可知\exists F'(\zeta)=0$

$\nl$

$例:f(x)在[0,1]上可导,f(0)=0,f(1)=1$

$则对\forall \lambda_k>0(k=1,2,3...n)必存在x_k \in (0,1) (k=1~n)$

$使得\frac{\lambda_1}{f'(x_1)}+\frac{\lambda_2}{f'(x_2)}+...\frac{\lambda_n}{f'(x_n)}=\lambda_1+\lambda_2+...\lambda_n$

$证明n=2的情形$

$f(0)=0,f(1)=1, \exists x_0,0<f(x_0)=\frac{\lambda_1}{\lambda_1+\lambda_2}<1)$

$f(x_0)-f(0)=f'(x_1)x_0$

$f(1)-f(x_0)=f'(x_2)(1-x_0)$

$x_0=\frac{f(x_0)}{f'(x_1)}=\frac{\frac{\lambda_1}{\lambda_1+\lambda_2}}{f'(x_1)},1-x_0=\frac{1-f(x_0)}{f'(x_2)}=\frac{\frac{\lambda_2}{\lambda_1+\lambda_2}}{f'(x_2)}$

$\nl$

$f(x)在[-1,1]上无穷次可导,且f(\frac{1}{n})=\frac{n^2}{1+n^2},n\in N$

$求f^{(n)}(0),令g(x)=\frac{1}{1+x^2},F(x)=f(x)-g(x),\forall x \in [-1,1]$

$F(\frac{1}{n})=f(\frac{1}{n})-g(\frac{1}{n})=0,故F'(0)=0,F''(0)=0,....$

$\nl$

$例:f(x)在(0,a]上可导,且\xtx{0^+}\sqrt{x}f'(x)存在$

$证明:f(x)在(0,a]上一致连续$

$\forall M>0,\exists \delta >0,|\sqrt{x}f'(x)| \le M. 0<x<\delta$

$|\frac{f'(x)}{\frac{1}{\sqrt{x}}}| \le M, g(x)=2\sqrt{x}, \forall x_1,x_2 \in (0,\delta)$

$用Cauchy,|\frac{f(x_1)-f(x_2)}{g(x_1)-g(x_2)}|=|\frac{f'(\zeta)}{g'(\zeta)}|=|\sqrt{\zeta}f'(\zeta)| \le M$

$取\forall \epsilon>0,取\delta_2=\frac{\epsilon^2}{16M}>0$

$|\sqrt{x_1}-\sqrt{x_2}|<\sqrt{x_1}+\sqrt{x_2}<\frac{\epsilon}{2M}$

$\delta = min(\delta_1,\delta_2)$

$\nl$

$例:Jensen不等式,设f(x)为[a,b]上凸函数,x_i \in [a,b]$

$\lambda_i>0,(i=1~n)且\sum_{i=1}^n \lambda_i=1,则有$

$f(\lambda_1x_1+\lambda_2x_2+...\lambda_nx_n) \le \lambda_1f(x_1)+...\lambda_nf(x_n)$

$\lambda=2时由定义可知成立,设n-1时成立$

$t=\sum_{i=2}^n \lambda_i,\frac{\sum_{i=2}^n \lambda_i}{t}=1,\lambda_1 +t=1$

$\sum_{i=2}^n \frac{\lambda_i}{t} x_i \le max_{i=2~n}(x_i) \to \sum_{i=2}^n \frac{\lambda_i}{t} x_i \in [a,b]$

$f(\sum_{i=1}^n \lambda_i x_i) =f(\lambda_1x_1+(\sum_{i=2}^n \frac{\lambda_i}{t} x_i)t)$

$\le \lambda_1f(x_1)+tf(\sum_{i=2}^n \frac{\lambda_i}{t} x_i)$

$\le \lambda_1f(x_1)+t \sum_{i=2}^n \frac{\lambda_i}{t} f(x_i)$

$=\sum_{i=1}^n \lambda_if(x_i)$

$\nl$

$Young不等式,设x,y \ge 0,p,q>0, \frac{1}{p}+\frac{1}{q}=1,则$

$有x^{\frac{1}{p}}y^{\frac{1}{q}} \le \frac{x}{p}+\frac{y}{q}$

$x,y>0时,即\frac{1}{p}lnx+\frac{1}{q}lny \le ln(\frac{x}{p}+\frac{y}{q})$

$设f(x)=ln(x),证明f(x)为上凸函数,再利用Jensen不等式$

$\nl$

$例:\frac{1}{n}\sum_{i=1}^n sini \le sin(\frac{1}{n}\sum_{i=1}^n x_i),0<x_1<x_2<...x_n< \pi$

$\prod_{i=1}^n x_i^{x_i} \ge (\prod_{i=1}^n x_i)^{\frac{1}{n}\sum_{i=1}^n x_i}$

\end{document}

