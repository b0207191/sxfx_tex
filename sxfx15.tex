\documentclass[12pt,a4paper]{article}
\usepackage{fancyhdr}
\usepackage{fontspec}
\usepackage{amsmath}
\usepackage{amssymb}
\usepackage{bm}
\usepackage{tikz}
\setmainfont{Microsoft YaHei}
\pagestyle{fancy}

\begin{document}

\fancyfoot[C]{by chinasjtu@msn.com }

\newcommand{\nl}{\newline}

\newcommand{\ntinf}{\lim\limits_{n \to \infty}}
\newcommand{\xtinf}{\lim\limits_{x \to \infty}}

\newcommand{\Atinf}{\lim\limits_{A \to \infty}}
\newcommand{\Rtinf}{\lim\limits_{R \to \infty}}

\newcommand{\ntx}[1]{\lim\limits_{n \to #1}}
\newcommand{\xtx}[1]{\lim\limits_{x \to #1}}
\newcommand{\ttx}[1]{\lim\limits_{t \to #1}} 
\newcommand{\ktx}[1]{\lim\limits_{k \to #1}} 
\newcommand{\dxtx}[1]{\lim\limits_{\Delta x \to #1}}

\newcommand{\jfab}{\int_{a}^{b}}
\newcommand{\jf}[2]{\int_{#1}^{#2}}

\newcommand{\nsum}[2]{\sum\limits_{n=#1}^{#2}}
\newcommand{\isum}[2]{\sum\limits_{i=#1}^{#2}}
\newcommand{\ksum}[2]{\sum\limits_{k=#1}^{#2}}

\newcommand{\nsuminf} {\nsum{1}{\infty}}
\newcommand{\ksuminf} {\ksum{1}{\infty}}
\newcommand{\isuminf} {\isum{1}{\infty}}





\begin{center} 第15章 隐函数  \end{center}


$15.1(P206)$

$F(xy,y+z,xz)=0,(F_2+xF_3 \ne 0)求\frac{\partial z}{\partial x},\frac{\partial ^2z}{\partial x \partial y}$

$\frac{\partial z}{\partial x}:yF_1+F_2 \frac{\partial z}{\partial x} + (z+\frac{\partial z}{\partial x} x)F_3=0 \to .....$

$\frac{\partial ^2z}{\partial x \partial y}:F_1+x\{yF_1+f_2 \frac{\partial z}{\partial x} +(z+\frac{\partial z}{\partial x} x)F_{31}\}$

$+\frac{\partial ^2z}{\partial x \partial y} F_2+(1+\frac{\partial z}{\partial y})\{yF_{12}+\frac{\partial z}{\partial x}F_{22}+(z+\frac{\partial z}{\partial x})F_{32}\}$

$+(\frac{\partial z}{\partial y}+x \frac{\partial ^2z}{\partial x \partial y})F_3+x\frac{\partial z}{\partial y}\{yF_{13}+\frac{\partial z}{\partial x}F_{23}+(z+\frac{\partial z}{\partial x})F_{33}\}=0$

$将\frac{\partial z}{\partial x},\frac{\partial z}{\partial y}代入,\frac{\partial ^2z}{\partial x \partial y}=...$

$\nl$

$例:x^2+y+sinxy=0(1)证明:在(0,0)领域内可唯一确定连续函数y=y(x),y(0)=0$

$(2)讨论y=y(x)在x=0附近可导性$

$(3)讨论y=y(x)在x=0附近单调性$

$(4)在(0,0)领域是否唯一确定x=x(y)$

$\nl$

$设F(x,y,u,v)=0,G(x,y,u,v)=0为定义在V \subset R^4的两个函数$

$若存在平面域D,使对\forall (x,y) \in D,有唯一(u,v) \subset J×K 使$

$满足F(x,y,u,v)=0,G=(x,y,u,v)=0称在D上定义了$

$u,v为x,y的隐函数组,记为u=f(x,y),v=g(x,y)定理1$

$(隐函数存在定理)设F(x,y,u,v)G(x,y,u,v)满足$

$(1)F(x_0,y_0,u_0,v_0)=0,G(x_0,y_0,u_0,v_0)=0$

$(2)在点(x_0,y_0,u_0,v_0)为中心某领域内F,G连续,且对各变元有连续偏导数$

$(3)在点(x_0,y_0,u_0,v_0)处Jacobi行列式\frac{\partial (F,G)}{\partial (u,v)}= \begin{vmatrix} F_u & F_v \\ G_u & G_v \end{vmatrix} \ne 0$

$则有1.在点(x_0,y_0,u_0,v_0)某领域,由F=0,G=0可得出隐函数被u=f(x,y)u=g(x,y)$

$2.隐函数组(2)在V(x_0,y_0)P' 内连续$

$3.隐函数组(2)有连续偏导数U_x,U_y,V_x,V_y$

$证明:不妨由1,设F_u \ne 0,则有u=\varnothing(x,y,z)代入G有\psi(x,y,v)=0$

$在(x_0,y_0,u_0),\psi_2=a_n \varnothing_v+G_v=\begin{vmatrix} F_u & F_v \\ G_u & G_v \end{vmatrix} /  F_n \ne 0 故v=g(x,y)$

$\frac{\partial u}{\partial x}=-\frac{\begin{vmatrix} F_x & F_y \\ G_x & G_y \end{vmatrix}}{\begin{vmatrix} F_u & F_v \\ G_u & G_v \end{vmatrix}} = - \frac{\frac{\partial(F,G)}{\partial(x,u)}}{\frac{\partial(F,G)}{\partial(u,v)}}$

$\frac{\partial v}{\partial x}= - \frac{\frac{\partial(F,G)}{\partial(u,x)}}{\frac{\partial(F,G)}{\partial(u,v)}}$

$定理2.(反函数组存在定理)设u=u(x,y)v=v(x,y)连续并对$

$各变元在(x_0,y_0)某领域内连续,且u_0=u(x_0,y_0)v_0=v(x_0,y_0)$

$以及\frac{\partial(u,v)}{\partial(x,y)}|_{(x_0,y_0)} \ne 0,则在点(u_0,v_0)某领域内存在有连续偏导$

$的反函数组x=x(u,v),y=y(u,v)且有$

$\frac{\partial x}{\partial u} = \frac{\frac{\partial v}{\partial y}}{\frac{\partial(u,v)}{\partial(x,y)}}; \frac{\partial x}{\partial v}=-\frac{\frac{\partial u}{\partial y}}{\frac{\partial(u,v)}{\partial(x,y)}}$

$\frac{\partial y}{\partial u} = -\frac{\frac{\partial v}{\partial x}}{\frac{\partial(u,v)}{\partial(x,y)}}; \frac{\partial u}{\partial v}=\frac{\frac{\partial u}{\partial x}}{\frac{\partial(u,v)}{\partial(x,y)}}$

$\nl$

$例:设\begin{cases} x+y+z+u+v=1 \\ x^2+y^2+z^2+u^2+v^2=2 \end{cases}求u_x,v_x,u_{x^2},v_{x^2}$

$分析:改写为F(x,y,z,u,v)=x+y+z+u+v-1=0$

$G=x^2+y^2+z^2+u^2+v^2-2=0$

$\frac{\partial(F,G)}{\partial(u,v)}=\begin{vmatrix} 1 & 1 \\ 2u & 2v \end{vmatrix}=2(v-u)$

$当u-v \ne 0时,可确定u=u(x,y,z),v=v(x,y,z)$

$(1)对x求偏导,1+\frac{\partial u}{\partial x}+\frac{\partial v}{\partial x}=0$

$2x+2u \frac{\partial u}{\partial x}+ \frac{\partial v}{\partial x} 2v =0$

$解得\frac{\partial v}{\partial x}=\frac{u-x}{v-u};\frac{\partial u}{\partial x}=\frac{x-v}{v-u}$

$(2)再对x偏导,\frac{\partial ^2u}{\partial x^2}+\frac{\partial ^2v}{\partial x^2}=0$

$1+(\frac{\partial u}{\partial x})^2+u \frac{\partial ^2u}{\partial ^2x}+(\frac{\partial v}{\partial x})^2+u \frac{\partial ^2v}{\partial ^2x}=0$

$解出关于\frac{\partial u}{\partial x}的关系式,无须代入$

$\nl$

$例:x=cos \phi cos \psi, y=cos \varphi cos \psi, z=sin \varphi,求z_{x^2}$

$对x偏导,1=-sin \varphi cos \psi \frac{\partial \phi}{\partial x}-cos \varphi sin \psi \frac{\partial \psi}{\partial x}$

$0=-sin \varphi sin \psi \frac{\partial \phi}{\partial x}+cos \varphi cos \psi \frac{\partial \psi}{\partial x}$

$得出\frac{\partial \varphi}{\partial x}=-\frac{cos \psi}{sin \varphi} ,\frac{\partial \psi}{\partial x}=-\frac{sin \psi}{cos \varphi}$

$\frac{\partial z}{\partial x}=cos \varphi \frac{\partial \varphi}{\partial x} = -ctg\varphi cos \psi$

$\frac{\partial ^2z}{\partial x^2}=\frac{\partial}{\partial \varphi}(\frac{\partial z}{\partial x})\frac{\partial \varphi}{\partial x}+\frac{\partial}{\partial \psi}(\frac{\partial z}{\partial x})\frac{\partial \psi}{\partial x}$

$另解:(x^2+y^2+z^2=1 \to ....)$

$\nl$

$例:设u+v=x+y,\frac{sinu}{sinv}=\frac{x}{y},求du,dv,d^2u,d^2v$

$微分,du+dv=dx+dy(*),sinudy+cosu y du =dx sinv+cosv xdv(**)$

$解出dv,du;再对(*)微分,d^2u+d^2v=0(dx,dy作为常数)$

$(**)微分,ycosud^2u+2cosudydu-ysiudu^2$

$=xcosvd^2v+2cosvdxdv-xsinvdv^2$

$\nl$
$\nl$

$函数行列式及其应用$

$Jacobi矩阵(P211)(P216) \frac{dy}{dt}=\frac{dy}{dx} \frac{dx}{dt}$

$球坐标交换的Jacobi行列式$

$x=rsin\varphi cos \theta,y=rsin\varphi sin\theta ,z=rcos\varphi$

$\frac{\partial (x,y,z)}{\partial (r,\varphi,\theta)}=r^2 sin \varphi$

$命题:设有定义在D \subset R^n上的函数组,且对每个f_k(1 \sim m)对各变元有连续偏导$

$其Jacobi矩阵为D(x)=\begin{vmatrix} \frac{\partial f_1}{\partial x_1} & ... & \frac{\partial f_1}{\partial x_n} \\ ... & ... & .... \\ \frac{\partial f_m}{\partial x_1} & ... & \frac{\partial f_m}{\partial x_n} \end{vmatrix}_{m*n}(m \le n)$

$则矩阵r(J)<m,函数相关(P222)$

$当m=n r(J)=n,函数线性无关$

$\nl$

$例:设x,y \ge 0,证明 \frac{x^n+y^n}{2} \ge (\frac{x+y}{2})^n$

$设x+y=a,则L=\frac{1}{2}(x^n+y^n)+\lambda (x+y-a)$

$由L_x=0,L_y=0,x+y=a解出x=y=\frac{a}{2}与边界(0,a)(a,0)比较$

$条件极值的充分条件,H矩阵判定法$

$H=\bigl( \begin{smallmatrix} L_{x^2}&L_{xy}&L_{xz}\\ L_{yx}&L_{y^2}&L_{yz} \\ L_{zx}&L_{zy}&L_{z^2} \end{smallmatrix} \bigr) $

$正定极小,负定极大,不定$

$\nl$

$二阶微分判别法(极值性)$

$求f=x+y+z+t在xyzt=c^4下极值(见P201)$

$解:L=f+\lambda(xyzt-c^4)得x=y=z=t=c,\lambda = -\frac{1}{c^3}$

$dL=-\frac{1}{c^3}(yztdx+xztdy+xytdz+xyzdt)$

$d^2L|_{P_0}=-\frac{2}{c}(dxdy+dydz+dxdz+dt(dx+dy+dz))$

$对xyzt=c^4求微分$

$dx+dy+dz+dt=0 \to dt=-(dx+dy+dz)$

$解方程组$

$\begin{cases} 2x+2\lambda x+2u=0 \\ 2y+2\lambda y-u =0 \\ 2z-2\lambda z -u=0 \\ x^2+y^2-z^2=1 \\ 2x-y-z=1 \end{cases}$

$\lambda = \pm 1 代入解方程,当\lambda \ne \pm 1时,前三式行列式系数不为0$

$解出x=\frac{-u}{1+\lambda},y=\frac{u}{2(1+\lambda)},z=\frac{u}{2(1-\lambda)}代入后两式$

$3\lambda ^2 -9 \lambda + 8 =0$

$\nl$

$例:x^2+y+sin(xy)=0$

$证明在(0,0)某邻域内唯一连续函数y=y(x)使y(0)=0(F_y(0,0)=1)$

$y(x)在x=0附近可导(F_x为连续函数)$

$y(x)在x=0附近单调性y'=- \frac{2x+ycos(xy)}{1+xcos(xy)}$

$y'(0)=0故有\xtinf \frac{y}{x}=0,y=o(x)$

$\xtx{0}=\frac{-\frac{2x+ycosxy}{1+xcosxy}}{-2x}=\xtx{0}\frac{1+o(1)\frac{cosxy}{2}}{1+xcosxy}=1$

$故不能确定x=x(y)$

$\nl$

$p.s. 平面既是开区域也是闭区域(唯一特例)$

$偏导存在 \nleftarrow 方向导数存在,z=\sqrt{x^2+y^2}$

$偏导存在 \nrightarrow 方向导数存在,f=\begin{cases} \frac{xy}{\sqrt{x^2+y^2}}sin\frac{1}{x^2+y^2}, & x^2+y^2 \ne 0 \\ 0, & x^2+y^2=0 \end{cases}$

$若f(y,x)=-f(x,y),f''_{yx}=-f''_{xy}$

$若f(x,y)=A(x)B(y)则麦克劳林公式展开可用M(x,y)=M(x)M(y)略去高阶无穷小$

$\\$

$\nl$

含参数积分

$(P242)I=\jf{0}{+\infty} \frac{sinx}{x}dx,引进e^{-\alpha x},I(\alpha)=\jf{0}{+\infty}e^{-\alpha x}\frac{sinx}{x}dx(\alpha \ge 0)$

$计算\aatx{o^+}I(\alpha),I'(\alpha)=-\jf{0}{+\infty}e^{-\alpha x}sinxdx$

$故I(\alpha)=-arctg \alpha +C=\frac{e^{-\alpha x}(\alpha sinx+cosx)}{1+\alpha^2}|_0^{+\infty}=-\frac{1}{1+\alpha^2}$

$|I(\alpha)| \le \jf{0}{+\infty}e^{-\alpha x}dx=\frac{1}{\alpha},故\aatx{\infty}I(\alpha)=0$

$故C=\frac{\pi}{2},故I=I(0)=\frac{\pi}{2}$

$\nl$

$\jf{0}{\frac{\pi}{2}}lnsintdt=-\frac{\pi}{2}ln2$

$\nl$

$积分式:\frac{arctgx}{x}=\jf{0}{1}\frac{dy}{1+x^2y^2};\frac{e^{-ax}-e^{-bx}}{x}=\jf{a}{b}e^{-xy}dy$

$\frac{x^b-x^a}{lnx}=\jf{a}{b}x^ydy;\frac{ln(1+ab)}{a}=\jf{0}{b}\frac{dx}{1+ax}$

$\nl$

$称\jf{\alpha}{+\infty}f(x,y)dy关于x在I上不一致收敛,是指\exists \epsilon_0 >0以及一递$

$增无界数列\{A_n\}(A_n \ge \alpha)及\{x_n\}\subset I:|\jf{A_n}{+\infty}f(x_n,y)dy| \ge \epsilon_0$

$证明:\jf{0}{+\infty}ye^{-xy}dx在(y_0,+\infty)一致收敛,在(0,+\infty)不一致收敛$

$1,|\jf{A_0}{+\infty}ye^{-xy}dx|=|(-e^{xy})|_{x=A_0}^{x=+\infty}|=e^{-yA_0} \le e^{-y_0A_0}$

$故\forall \epsilon > 0 (0<\epsilon <1)取A_0= A_0(\epsilon)=\frac{ln \frac{1}{\epsilon}}{y_0},则对\forall y \in (y,+\infty)$

$|\jf{A}{+\infty}ye^{-xy}dx|<e^{-yA}<e^{k\epsilon}=\epsilon,\forall A>A_0$

$2,A_k=k,y_k=\frac{1}{k}\in (0,+\infty)则有\jf{k}{+\infty}\frac{e^{-\frac{x}{k}}}{k}dx=\frac{1}{e},\forall k \in N$

$取\epsilon_0 > \frac{1}{e},则有....或者说,当y充分小时,I>\epsilon_0=...$

$\nl$

$Cauchy准则,\jf{\alpha}{+\infty}f(x,y)dx一致收敛 \Leftrightarrow \forall \epsilon>0,\exists A \ge \alpha, \forall A',A'' \ge A 有$

$|\jf{A'}{A''}fdy|<\epsilon ,\forall x \in I$

$\nl$

$Abel推论,设\jf{\alpha}{+\infty}fdy关于x一致收敛,h(y)在(\alpha,+\infty)上单调有界,则\jf{\alpha}{+\infty}fhdy一致收敛$

$设\jf{\alpha}{+\infty}f(y)dy收敛,g(x,y)对y单调,且对x一致有界,则\jf{\alpha}{+\infty}f(y)g(x,y)dy一致收敛$

$\nl$

$例:\jf{0}{+\infty}\frac{sinx^2}{1+x^n}dx,解:0<\frac{1}{1+x^n} \le 1单调递减,且\jf{0}{+\infty}sin(x^2)dx一致收敛$

$\nl$

$例:已知\jf{-\infty}{+\infty}e^{-x^2}dx=\sqrt{\pi},判别\jf{-\infty}{+\infty}e^{-(x-y)^2}dx在\begin{cases} 
a<y<b \\
y\in R
\end{cases}$

$\jf{A}{+\infty}e^{-(x-y)^2}dx=\jf{A-y}{+\infty}e^{-x^2}dx < \jf{A-b}{+\infty}e^{-x^2}dx$

$故\forall \epsilon > 0,\exists A, \forall y<b,有...类似 \forall y>a有....取公共区间$

$再求y\in R,\forall A>0,充分大\infty,有y充分大 \jf{A}{+\infty}e^{-(x-y)^2}dx=\jf{A-y}{+\infty}e^{-x^2}dx \to \sqrt \pi$

$\nl$

$例:\jf{0}{+\infty} \frac{cos(xy^2)}{x^ \alpha}dx,y \ge y_0 > 0,(0< \alpha <1)的一致收敛性$

$\forall A>1 有|\jf{1}{A}cos(xy^2)dx|=\frac{1}{y^2},|sinAy^2-siny^2| \le \frac{2}{y^2} \le \frac{2}{y_0^2}$

$又因为\frac{1}{x^ \alpha}对x单调,一致趋0,由狄尼克莱I_2=\jf{1}{+\infty}...一致收敛$

$由|\frac{cos(xy^2)}{x^ \alpha}|\le \frac{1}{x ^ \alpha},\forall x \forall y ...而\jf{0}{1}\frac{1}{x^ \alpha}dx一致收敛$

$f(x,y) \in C[a,b] \times [c,+\infty),\jf{c}{+\infty}f(x,y)dy在(a,b)收敛,在a发散$

$则\jf{c}{\infty}f(x,y)dy在(a,b)不一致收敛$

$\nl$

$f(x,y)在D非负连续,若\jf{\alpha}{+\infty}fdy,\jf{a}{+\infty}fdx 分别为xy连续函数,且$

$\jf{a}{+\infty}dx\jf{\alpha}{+\infty}fdy或\jf{\alpha}{+\infty}dy\jf{a}{+\infty}fdx至少有一项存在,则积分可换序$

$\jf{0}{+\infty}\frac{e^{-xa}-e^{-xb}}{x}dx(a,b>0),1.f(x,y)=e^{-xy} \in c,2. \jf{0}{+\infty}e^{-yx}一致收敛$

$故有原式=\jf{a}{b}dy(\jf{0}{+\infty}e^{-xy}dx)=\jf{a}{b}\frac{1}{y}dy$

$\nl$

$计算I=\jf{0}{+\infty}e^{-x^2}dx$

$令x=ut(u>0),I=u\jf{0}{+\infty}e^{-u^2t^2}dt,同乘e^{-u^2}再对u积分$

$I^2=I\jf{0}{+\infty}e^{-u^2}du=\jf{0}{+\infty}e^{-u^2}udu\jf{0}{+\infty}e^{-u^2t^2}dt(先t后u)$

$验证(i)ue^{-u^2(1+t^2)}对tu非负连续(u>0)$

$(ii)\jf{0}{+\infty}ue^{-u^2(1+t^2)}dt=e^{-u^2}\jf{0}{+\infty}e^{-(ut)^2}dut=e^{-u^2}I在u\ge0时连续$

$\jf{0}{+\infty}ue^{-u^2(1+t^2)}du=\frac{1}{2} \frac{1}{1+t^2}在t \ge 0 时连续$

$(iii)\jf{0}{+\infty}dt\jf{0}{+\infty}ue^{-u^2(1+t^2)}du存在$

$故I^2=\frac{\pi}{4},I=\frac{\sqrt \pi}{2}$

$\jf{0}{+\infty}\frac{e^{-ax}-e^{-bx}}{x}dx,令I(a)=\jf{0}{+\infty}\frac{e^{-ax}-e^{-bx}}{x}dx \triangleq \jf{0}{+\infty}f(x,a)dx$

$f(0,a)=b-a,1.f与f_a \in D,2.\jf{0}{+\infty}\frac{e^{-ax}-e^{-bx}}{x}dx 收敛(Dirichlet)$

$\jf{0}{+\infty}-e^{-ax}dx在(a,+\infty)关于a一致收敛I'(a)=-\frac{1}{a}$

$\to I(a)=-lna+c令a=b \to c=lnb,I(a)=ln\frac{b}{a}$

$\nl$

$其他积分法①B(p,q)=2\jf{0}{\frac{\pi}{2}}cos^{2p-1}\varphi sin^{2q-1}\varphi d\varphi$

$B(\frac{1}{2},\frac{1}{2})=\pi(x=cos^2 \varphi)$

$②B(p,q)=\jf{0}{+\infty}\frac{t^{q-1}}{(1+t)^{p+q}}dt(x=\frac{1}{1+t})$

$③定义\Gamma(s)=\frac{\Gamma(S+1)}{s},-1<s<0,类似地,。。。。-a-1<s<-a,除去整数点$

$④\Gamma(s)=a^s \jf{0}{+\infty}t^{s-1}e^{-at}dt(x=at)$

$⑤\Gamma(s)=2\jf{0}{+\infty}t^{2s-1}e^{-t^2}dt,x=t^2$

$\Gamma(\frac{1}{2})=\sqrt \pi , \Gamma(\frac{1}{2}+n)=\frac{(2n-1)!!}{2^n} \sqrt \pi$

应该避免形式运算

$(2)倍量公式s>0,\Gamma(2s)=\frac{2^{2s-1}}{\sqrt \pi}\Gamma(s)\Gamma(s+\frac{1}{2})$

$(3)余元公式(0<s<1)$

$\Gamma(s)\Gamma(1-s)=\frac{\pi}{sin\pi s}$

$B(s,1-s)=\frac{\pi}{sin\pi s}$

$n! = (\frac{n}{e})^n \sqrt {2\pi}n e^{\sqrt 2^\theta n}$

$=(\frac{n}{e})^n \sqrt {2\pi}n(1+o(1)) (n \to \infty)$

$\nl$

$例:(1)\jf{0}{+\infty}x^me^{-ax^n}(m>-1,n>0,a>0)$

$令y=ax^n,则=\jf{0}{+\infty}(\frac{y}{a})^{\frac{m}{n}}e^{-y}\frac{1}{na}(\frac{y}{a})^{\frac{1}{n}-1}dy$

$=\frac{\Gamma(\frac{m+1}{n})}{na^{\frac{m+1}{n}}}$

$特别的,\jf{0}{+\infty}x^{2m}e^{-x^2}dx=\frac{(2n-1)!!}{2^{n+1}} \sqrt n$

$jf{0}{+\infty}e^{-x^n}dx=\frac{1}{n}\Gamma(\frac{1}{n})$

$\jf{0}{+\infty}e^{-y^4}dy \jf{0}{+\infty}x^2 e^{-x^4}dx=\frac{1}{16}\Gamma(\frac{1}{4})\Gamma(\frac{3}{4})= \frac{1}{16} \frac{\pi}{sin \frac{\pi}{4}}$

$(2)\jf{0}{1}x^{p-1}(1-x^m)^{q-1}dx(p,q,m>0)$

$令t=x^m即有=\frac{1}{m}B(\frac{p}{m},q)$

$\jf{0}{1}\sqrt {x^3(1-\sqrt x)}dx = \jf{0}{1}x^{\frac{3}{2}}(1-x^{\frac{1}{2}})^{\frac{1}{2}}dx=2\frac{\Gamma(5)\Gamma(\frac{3}{2})}{\Gamma(5+\frac{3}{2})}$

$\nl$

$习题课$

$Dirichelet积分 I=\jf{0}{+\infty}\frac{sinx}{x}dx=\frac{\pi}{2}$

$\jf{-\infty}{+\infty}\frac{sinx}{x}cos \lambda x dx=(偶函数,积化和差)=\begin{cases} 0, & |\lambda|>1 \\ \frac{1}{2}, & |\lambda|=1 \\ 1,& |\lambda|<1 \end{cases}$

$(2)Fresnel积分,I=\jf{0}{+\infty}sin(x^2)dx$

$分析:令x^2=t,则化为\frac{1}{2}\jf{0}{+\infty}\frac{sint}{\sqrt t}dt=\frac{1}{2} \frac{2}{\sqrt \pi}\jf{0}{+\infty}sint dt \jf{0}{+\infty}e^{-tu^2}du$

$=\frac{1}{\sqrt \pi}\jf{0}{+\infty}du \jf{0}{+\infty}e^{-tu^2}sintdt= \frac{1}{\sqrt \pi}\jf{0}{+\infty}\frac{du}{1+u^4}=\frac{\pi}{2\sqrt 2}$

$引进e^{-\alpha t}(\alpha > 0),I(\alpha)=\frac{1}{2}\jf{0}{+\infty}\frac{sint}{\sqrt t}e^{-\alpha t}dt = \frac{1}{\sqrt \pi}\jf{0}{+\infty}du\jf{0}{+\infty}e^{-(\alpha+u^2)t}sintdt$

$\frac{1}{\sqrt \pi}\jf{0}{+\infty}\frac{du}{1+(\alpha+u^2)^u}$

$验证换序条件:1.对u取e^{-\alpha t}为优函数$

$对t取e^{-t_0 u^2}$

$2.绝对值积分:\frac{1}{\sqrt \pi}\jf{0}{+\infty}e^{-\alpha t}|sint|dt \jf{0}{+\infty}e^{-tu^2}du$

$=\jf{0}{+\infty}e^{-\alpha t}\frac{|sint|}{\sqrt t}dt由P-判别法可知右端收敛$

$3.因为\jf{0}{+\infty}\frac{du}{1+(\alpha - u^2)^2}关于\alpha 一致收敛在(0,+\infty)上,I(\alpha) \in [0,+\infty)$

$常用积分变换式$

$1.\frac{e^{-af(x)}-e^{-bf(x)}}{f(x)}=\jf{a}{b}e^{-f(x)y}dy$

$2.\frac{1}{f(x)}=\jf{0}{+\infty}e^{-f(x)y}dy,f(x)>0$

$3.\frac{1}{\sqrt {f(x)}}=\frac{2}{\sqrt \pi}$

$4.\frac{sinbf(x)-sinaf(x)}{f(x)} = \jf{a}{b}cos(f(x)y)dy$

$5 \frac{arctgbf(x)-arctgaf(x)}{f(x)} = \jf{a}{b} \frac{dy}{1+(f(x)y)^2}$

$例:设f(x,y) \in C[a,b]×[\alpha,+\infty),而\jf{\alpha}{+\infty}f(b,y)dy发散$

$证明\jf{\alpha}{+\infty}f(x,y)dy在[a,b)不一致收敛(端点处发散,区间内必发散)$

$分析:反证法,\forall \epsilon >0:\exists A_0: \forall A',A'' \ge A_)$

$I(x) \triangleq |\jf{A'}{A''}f(x,y)dy| < \epsilon,\forall x \in [a,b)$

$由条件f(x,y) \in C.... 故常义含参积分\jf{A'}{A''}f(x,y)dy 为x的连续函数$

$令x \to b 有 |\jf{A'}{A''}f(b,y)dy| \le \epsilon,从而由Cauchy准则可知...$

$\nl$

$例:设f(t)=\jf{1}{+\infty}\frac{cosxt}{1+x^2}dx$

$证明:①\ttinf f(t)=0,|\jf{1}{A_0}\frac{cosxt}{1+x^2}dx|+|\jf{A_0}{+\infty}\frac{cosxt}{1+x^2}dx| < \epsilon + \epsilon$

$(黎曼引理,f(x)可积,\lbdtx{\infty} \jf{a}{b} f(x)sin \lambda x dx =0, \lbdtx{\infty} \jf{a}{b} f(x)cos \lambda x dx =0)$

$②f(t)在[0,\pi]上至少有一个零点$

$因为f(0)=\frac{\pi}{4},f(\pi) \le 0,可证\jf{0}{\pi}f(t)dt \le 0,或\jf{0}{\pi}sintf(t)dt \le 0$

$积分顺序:可化为\jf{1}{+\infty}dx \jf{0}{\pi} \frac{cosxtsint}{1+x^2}dx$

$=\jf{1}{+\infty}\frac{1+cos\pi x}{1+x^4}dx \le 0$
\end{document}

