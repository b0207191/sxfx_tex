\documentclass[12pt,a4paper]{article}
\usepackage{fontspec}
\usepackage{amsmath}
\usepackage{amssymb}
\usepackage{bm}
\usepackage{tikz}
\setmainfont{Adobe Kaiti Std}
\thispagestyle{empty}
\pagestyle{empty}
\begin{document}

\newcommand{\ntinf}{\lim\limits_{n \to \infty}}
\newcommand{\ntx}[1]{\lim\limits_{n \to #1}}
\newcommand{\xtx}[1]{\lim\limits_{x \to #1}}
\newcommand{\dxtx}[1]{\lim\limits_{\Delta x \to #1}}

第三章 连续函数

P76

$设y=f(x),x \in U(x_0)。若x在x_0处改变量为\Delta x,则y的改变量\Delta y=f(x_0+\Delta x)-f(x_0)满足$

$\dxtx{0} \Delta y = \dxtx{0}[f(x_0+\Delta x)-f(x_0)]=0,称为f(x)在x_0连续$

p.s.极限证明问题

设$x_1=a,x_2=b,x_{n+1}=\frac{x_n+x_{n-1}}{2}$

证明:$\frac{x_{n+1}-x_n}{x_n-x_{n-1}}<0有单调性$

~
设$\ntinf \frac{a_1+a_2+...+a_n}{n}=a,证明:\ntinf \frac{a_n}{n}=0$

用Cauchy准则,由条件$\forall \varepsilon >0,\exists N_1 \in N$

$|\frac{a_1+a_2+...+a_n}{n}-\frac{a_1+a_2+...+a_n+a_{n+1}}{n+1}|<\varepsilon,\forall n > N$

$|\frac{a_1+a_2+...a_n+n·a_{n+1}}{n(n+1)}|<\varepsilon \Rightarrow |\frac{a_1+a_2+...a_n}{(n+1)n}-\frac{a_{n+1}}{n+1}|<\varepsilon$

于是有

$\frac{a_1+a_2+...a_n+a_{n+1}}{n(n+1)}-\varepsilon < \frac{a_{n+1}}{n+1} < \frac{a_1+a_2+...a_n+a_{n+1}}{n(n+1)} +\varepsilon$

$由\ntinf \frac{a_1+a_2+...+a_n}{n}=a,故对上述\varepsilon > 0,\exists N_2 \in N$

$a-\varepsilon <\frac{a_1+a_2+...+a_n}{n}<a+\varepsilon, \forall n>N_2 $

故有
$\frac{a-\varepsilon}{n+1}-\varepsilon<\frac{a_{n+1}}{n+1}<\frac{a+\varepsilon}{n+1}+\varepsilon$

因为
$\ntinf \frac{1}{n+1}=0,故\exists N_3 \in N,\frac{a-\varepsilon}{n+1}>-\varepsilon,\frac{a+\varepsilon}{n+1}<\varepsilon$

取$N=max(N_1,N_2,N_3)则有$

$-2\varepsilon<\frac{a_{n+1}}{n+1}<2\varepsilon,\forall n>N$

故$\ntinf \frac{a_n}{n}=0$


Cauchy收敛准则

$\xtx{\infty} f(x)存在 \Leftrightarrow \forall \varepsilon >0,\exists A>0,\forall x'>A,\forall x''>A有|f(x')-f(x'')|<\varepsilon$


方法2:浙大法

$\frac{a_n}{n}=\frac{a_1+a_2+...+a_n}{n}-\frac{a_1+a_2+...+a_{n-1}}{n-1}·\frac{n-1}{n}$

$同取lim,则有\ntinf \frac{a_n}{n}=a-a·1=0$

$\\$

Ex.2.
$(2-x_n)x_{n+1}=1$

$x_1 \ne 2, x_1 \ne 1$

$x_1<1 \to x_2<1 \to x_2>x_1$ (*)

$x_1>2 \to x_2<1 \to 同上*$ (**)

$1<x_1<2 \to x_2>x_1 \to x_2>2 \to 同(**)$

$\ \ \ \ \ \ \ \ \ \qquad \qquad \qquad \to x_2<2 \to x_3<2 \to x_n <2 则 \ntinf x_n \le 2$

方法2
$x_1 \ne 1,有\frac{1}{x_{n+1}-1}=\frac{1}{x_n-1}-1=....=\frac{1}{x_1-1}-n$

$推出x_{n+1}-1=\frac{1}{\frac{1}{x_1-1}-n} \to 0 (n \to \infty)$


二,stolze定理及其应用

定理1.
$设\{b_n\}是严格单调的递减数列,且\ntinf a_n=\ntinf b_n = 0$

则当$\ntinf \frac{a_n-a_{n+1}}{b_n-b{n+1}}存在(有限或无穷)时,\ntinf \frac{a_n}{b_n}也存在,且$

$\ntinf \frac{a_n}{b_n} = \ntinf \frac{a_n-a_{n+1}}{b_n-b{n+1}} $

定理2.

$设\{b_n\}为严格递增的数列,且\ntinf b_n = + \infty$

则当$\ntinf \frac{a_n-a_{n+1}}{b_n-b{n+1}}存在(有限或无穷)时,\ntinf \frac{a_n}{b_n}也存在,且$

$\ntinf \frac{a_n}{b_n} = \ntinf \frac{a_n-a_{n+1}}{b_n-b{n+1}} $


用法示例

(1)
$证明\ntinf a_n=a \to \ntinf \frac{a_1+a_2+....+a_n}{n}=a$

$令 \frac{x_n}{y_n}=\frac{a_1+a_2+....+a_n}{n}$

(2)
$\ntinf \frac {\sqrt{1}+\sqrt{2}+....+\sqrt{n}}{n \sqrt{n}}  $

$= \ntinf \frac{\sqrt{n+1}}{(n+1)\sqrt{n+1}-n\sqrt{n}}$

$= \ntinf \frac{\sqrt{n+1}[(n+1)\sqrt{n+1}+n\sqrt{n}]}{(n+1)^3-n^3}$

$= \frac{2}{3}$

(3)
$\ntinf \frac{ln n}{n}= \ntinf \frac{ln(n+1)-lnn}{n+1-n}=\ntinf ln(1+\frac{1}{n})=0$

(4)
$\ntinf \frac {n^2}{a^n} (a>1)= \ntinf \frac{(n+1)^2-n^2}{a^{n+1}-a^n} =$

$\ntinf \frac {2n+1}{a^n(a-1)} = \frac{1}{a-1} \ntinf \frac{2}{a^{n+1}-a^n}=0$

(5)
$0<x_n<1, x_{n+1}=x_n(1-x_n)$

求证$\ntinf nx_n=1$

证明:

1.证明$x_n递减有界,\ntinf x_n =0$

2.$\{ \frac{1}{x_n} \} 严格递增,极限 \to + \infty$

3.$nx_n = \frac{n}{\frac{1}{x_n}} = (stolze) \frac{1}{\frac{1}{x_n}-\frac{1}{x_{n-1}}}$

$=\frac{x_nx_{n-1}}{x_{n-1}-x_n}= \frac{x_n}{x_{n-1}}=1-x_{n-1} \to 1$


(6)
$x_1=sinx,x_{n+1}=sinx_n$

$\ntinf \sqrt{\frac{n}{3}}sinx_n=1,即nsin^2x_n=3$

$利用(\xtx{0}\frac{x^2sinx^2}{x^2-sinx^2}=3)$

续连续函数:P76
注意3项

$x \to x_0,f(x)有极限,区别:x=x_0处未必有定义$

$f(x)在x=x_0处连续$

$|f(x)-A|<\varepsilon$

$|f(x)-f(x_0)|<\varepsilon$

f(x)在a连续等价于

(1)$\xtx{a}f(x)=f(a)$

(2)$\xtx{a+}f(x)=\xtx{a-}f(x)=f(a)$

定义2,设f(x)在$x_0$的某左(右)领域有定义,且
$\xtx{x_0-}f(x)=f(x_0)(或\xtx{x_0+}f(x)=f(x_0))称f(x)在x_0处左(右连续)$

定理1

f(x)在点$x_0$处连续充要条件是:f(x)在点$x_0$处左连续又右连续

符号表示为

$(a,b)内,f(x) \in C(a,b)$

$[a,b]内,f(x) \in C[a,b]$

$R上,f(x) \in CR$


Ex(1)
$\xtx{0-}f(x)=+\infty$

$\xtx{0+}f(x)=0$

故f在$x_0$整体不连续

一致函数反函数一般不再一致连续

(2)
$y=\frac{1}{x}-[\frac{1}{x}]$

(3)
$
f(x)=\begin{cases}
sin\pi x, & x\in Q \\
0, & x\notin Q
\end{cases}
$

f(x)在Z连续,在非Z不连续

$f(x)在x_0连续 \ne f(x)在x_0附近连续$

二,间断点的分类,P83

$f(x)=\frac{cos\frac{\pi}{2}x}{x^2(x-1)}$
x=0为二类点
$x=1,f(x)=-\frac{\pi}{2}$

单调函数只可能有第一类间断点

定理1.

$设u=g(x)在x_0处连续,u_0=g(x_0),又有y=f(u)在u_0处连续,则复合函数f(g(x))在x_0处连续$

$由y=f(u_0)在u_0处连续,\forall \varepsilon >0,\exists \eta >0$

$|f(u)-f(u_0)|<\varepsilon, \forall u(|u-u_0|<\eta)$

$又u_0=g(x_0)及u=g(x)在x_0处连续$

$对上述\eta>0,\exists \delta>0$

$|u-u_0|=|g(x)-g(x_0)|<\eta,\forall x(|x-x_0|<\varepsilon)$

此时有
$|f(g(x))-f(g(x_0))|<\varepsilon,\forall x(|x-x_0|<\varepsilon)$


四,函数一致连续,P86(必须在区间上,公用的$\delta$)

$f(x)=x,x \in R, \forall x_0 \in R, \forall \varepsilon>0, \exists \delta=\varepsilon$

$|f(x)-f(x_0)|=|x-x_0|<\varepsilon,\forall x(|x-x_0|<\delta)$


$f(x)=x^2,x \in R, \forall x_0 \in R$

$|f(x)-f(x_0)|=|x-x_0|·|x+x_0|见P78$

数列型反例
$x_1=n,x_2=n+\frac{1}{n}$

一致连续等价叙述

$\{x_n\}\{y_n\} \in I, \ntinf (x_n)-(y_n)=0,有\ntinf f(x_n)-f(y_n)=0$

sinx与x在R上一致连续,但sinx·x在R上非一致连续

反例
$x_1=2n\pi+\frac{\delta}{3},x_2=2n\pi-\frac{\delta}{3}$

注:f(x)在I上不一致连续,是指$\exists \varepsilon_0 >0, \forall \delta >0$

$|f(x_1)-f(x_2)| \ge \varepsilon_0,当x_1,x_2 \in I,且|x_1-x_2|<\delta$

或即
$\exists \varepsilon_0 >0和 |x_n^{(1)}|,|x_n^{(2)}| \subset I,(\ntinf x_n^{(1)}-x_n^{(2)}=0)$

$|f(x_n^{(1)})-f(x_n^{(2)})| \ge \varepsilon_0$

先找到不一致连续之处,很陡时,$x_n \to 它$

Ex:
$(1)f(x)=sin\frac{1}{x},x \in(0,1)$

$x_n^{(1)}=\frac{1}{2n\pi},x_n^{(2)}=\frac{1}{2n\pi+\frac{\pi}{2}}$

$(2)f(x)=lnx, x \in (0,+\infty)$

$x_n^{(1)}=e^{-n},x_n^{(2)}=e^{-n-1}$

$\\$

五,闭区间上连续函数性质

定理3(最值定理P85)推论:有界

例6(P90)设
$f(x) \in C[a,+\infty),\xtx{+\infty}f(x)存在,证明f(x)在[a,+\infty)上有界$

$证明:设\xtx{\infty}f(x)=l,令\varepsilon_0=1,必存在A>a,当x>A时$

$|f(x)-l|<1 \to |f(x)<1+|l|$

$又f(x) \in C[a,A],从而有界,记|f(x)|\le M,\forall x \in [a,A]$

$|f(x)| \le max(M,1+|l|)$

定理4(零点存在,P85)

$x^3+4x^2-3x-1=0$

$+(0)=-1,f(1)=1,f(-1)=5$

$\xtx{-\infty}f(x) \to -\infty,由局部有界性可知,存在充分大大A>0$

$使f(-A)<0,故有第3根$

介值定理

$Ex:f(x) \in C[a,b],且x_1,x_2,..x_n \in [a,b]$

$存在\zeta \in [a,b],有f(\zeta)=\frac{f(x_1)+f(x_2)+...f(x_n)}{n}$

康托定理P115

在(a,b)内连续的函数f(x)在(a,b)内一致连续的充要条件是

$\xtx{a^{+0}}f(x)和\xtx{b^{-0}}f(x)均存在$

1.必要性

用柯西收敛准则

有定义可知
$\forall \varepsilon >0,\exists \delta >0,(0<\delta<b-a)$

$|f(x')-f(x'')|<\varepsilon,\forall x',x'' \in (a,b),且|x'-x''|<\delta$

$考虑(a,a+\delta)内任意两点x',x'',也即有$

$a<x'<a+\delta$

$a<x''<a+\delta$

此时,也有
$|x'-x''|<\delta,从而|f(x')-f(x'')|<\varepsilon$

由cauchy准则可知
$\xtx{a^{+0}}f(x)存在$

充分性。
作辅助函数
$
F(x)=\begin{cases}
l_1,& x=a \\
f(x), & a<x<b \\
l_2, & x=b
\end{cases}
$

故F(x)在[a,b]连续,即一致连续

此命题用于判断函数是否一致连续

例10,P117(*)

证明:
方法1:
$因f(x) \in C[a,M],f(x)在[a,M]上一致连续$

$对上述\varepsilon > 0,\exists \delta>0,当x',x'' \in [a,M],且|x'-x''|<\delta 时$

$有|f(x')-f(x'')|<\frac{\varepsilon}{2}$

当$x' \in [a,M],x'' \in [M,+\infty)且|x'-x''|<\delta ,$

$因为|x'-M|\le|x'-x''|<\delta,而x'' \ge M$

应有
$|f(x')-f(x'')|\le|f(x')-f(M)|+|f(x'')-f(M)|<\frac{\varepsilon}{2}+\frac{\varepsilon}{2}=\varepsilon$

方法2:
因
$\xtx{+\infty}f(x)存在,由柯西准则可知$

$\forall \varepsilon>0,\exists M>a:当x',x'' \ge M时,有|f(x')-f(x'')|<\frac{\varepsilon}{2}$

$f(x) \in C[a,M+1],从而f(x)在[a,M+1]上一致连续$

$对上述\varepsilon > 0,\exists \delta (0<\delta<1)$

使对$\forall x',x'' \in [a,M+1],当|x'-x''|<\delta 时$

$有|f(x')-f(x'')|<\frac{\varepsilon}{2}$

习题课
(1)构造函数f(x)

1.在R上处处不连续:D(x)

2.仅在0处连续

$
f(x)=\begin{cases}
x, & 有理数 \\
-x, & 无理数
\end{cases}
$

3.在[0,1]上任一无理点连续,任一有理点不连续:黎曼函数

4.在[-1,1]上不连续,但可取到最大值,最小值,一切中间值

5.在[-1,1]上处处不连续,但可取到最大值,最小值,一切中间值

对2函数改造

\begin{tikzpicture}
\draw[->] (-1.2,0) -- (1.2,0);
\draw[->] (0,-1.2) -- (0,1.2);
\draw[style=dashed] (-1,-1) -- (1,1);
\draw[style=dashed] (-1,0) -- (0,1);
\draw[style=dashed] (0,-1) -- (1,0);
\node [right] at (1.2,0) {x};
\node [above] at (0,1.2) {y};
\end{tikzpicture}


$设f(x)在u_0处连续,u=g(x)(u+0=g(x_0)在x \to x_0时$

$极限存在,是否必有f(g(x))在x=x_0处连续?$

考虑
$
f(u)=\begin{cases}
usin\frac{1}{u}, & u \ne 0 \\
0, & u=0
\end{cases}
g(x)=\begin{cases}
1, & x \ne 0 \\
0, & x=0
\end{cases}
$

$在x_0=0处不连续$


$例,设f(x)在R上满足lipshity条件$

$\forall x,y \in R,有|f(x)-f(y)| \le L·|x-y| (0<L<1)$

证明f(x)在R上有唯一不动点,也即有
$f(x_0)=x_0$

$构造数列x_{n+1}=f(x_n)$

$|x_{n+1}-x_n| \le L^{n-2}|x_2-x_1|$

$用Cauchy可证x_n收敛,记\xtx{\infty}x_n=x_0$

$x_0=\ntinf x_{n+1}= \ntinf f(x_n)=f(\ntinf x_n)=f(x_0)$

唯一性

$|x_0'-x_0|=|f(x_0')-f(x_0)| \le L·|x_0'-x_0|$ 

例:设
$f(x) \in C(a,b)且 \xtx{a+}f(x)=\xtx{b-}f(x)= -\infty$

证明:f(x)在(a,b)内必可取到最大值

$取[a+\delta,b-\delta]内最值f(c),\exists \delta>0 (a+\delta<\frac{a+b}{2})$

$f(x)<f(\frac{a+b}{2}),\forall x (a<x<a+\delta _1)$

例:
$f(x) \in C(R)为周期T的周期函数,证明f(x)在R上一致连续(非定理)$

$因f(x)\in C(R),故f(x)在[-T,T]上一致连续,\forall \varepsilon >0,$

$\exists \delta >0,(0<\delta <T) |f(y')-f(y'')|<\varepsilon,\forall y',y'' \in [-T,T]$

$|y'-y''|<\delta,\forall x',x'' \in R,|x'-x''|<R$

由f的周期性可知
$\exists y',y'' \in [-T,T]以及n \in Z$

$使得x'=nT+y',x''=nT+y'',且|y'-y''|<\delta$

$于是|f(x')-f(x'')|=|f(nT+y')-f(nT+y'')|=|f(y')-f(y'')|<\varepsilon$

$f(x)在R上不一致连续 \to f(x)不是周期函数$

$\newline$

$例:(sinx)^3+sin(x^3)$

$x_n^1=\sqrt[3]{2n\pi+\frac{\pi}{2}},x_n^2=\sqrt[3]{2n\pi-\frac{\pi}{2}}$

$\newline$

$f(x)满足lipshity条件,则\frac{f(x)}{x}一致连续$

$证明:|\frac{f(x)}{x}-\frac{f(y)}{y}| \le \frac{1}{x}|f(x)-f(y)|+|f(y)|·|\frac{1}{x}-\frac{1}{y}|$

$\le \frac{|f(x)-f(y)|}{a}+\frac{|x-y|}{xy}(|f(a)|+k|y-a|)$

$\le \frac{1}{a}(2k+|f(a)|)·|x-y|$

$\newline$

$例:f(x) \in  C(0,+\infty), \forall x \ge 0,有\ntinf f(x+n)=0,证明\ntinf f(x)=0$

$证:由一致连续条件,\forall \varepsilon >0,\exists \delta >0: |f(x)-f(y)|<\frac{\varepsilon}{2}$

$\forall x,y \in (0,+\infty), |x-y|<\delta$

$由|f(x)| \le |f(x)-f(x_k+n)|+|f(x_k+n)|$

现将[0,1]区间m等分,分点为$0=x_0<x_1<x_2<...x_m=1$

$且\Delta = x_k-x_{k-1}<\delta (k=1~m)$

$现对\forall x \in [0,1], \exists x_k 使得 |x-x_k| \le \delta $

$对\forall x \in (1,+\infty),\exists n \in N,及其x_k \in [0,1]$

$使|x-(x_k+n)| \le \delta$

$由条件\ntinf f(x+n)=0,对上述\varepsilon >0,\exists N \in \bm N$

$|f(x_k+n)|<\frac{\varepsilon}{2}, n > N$

于是$当x>N+1时,\exists n > N 及其某一r_k$

$使得|x-(x_k+n)| \le \delta$

$f(x) \le \frac{\varepsilon}{2}+\frac{\varepsilon}{2}= \varepsilon$
\end{document}

