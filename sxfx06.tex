\documentclass[12pt,a4paper]{article}
\usepackage{fancyhdr}
\usepackage{fontspec}
\usepackage{amsmath}
\usepackage{amssymb}
\usepackage{bm}
\usepackage{tikz}
\usepackage{pstricks-add}
\setmainfont{Microsoft YaHei}
\pagestyle{fancy}


\begin{document}

\fancyfoot[C]{by chinasjtu@msn.com }

\newcommand{\nl}{\newline}

\newcommand{\ntinf}{\lim\limits_{n \to \infty}}
\newcommand{\xtinf}{\lim\limits_{x \to \infty}}

\newcommand{\Atinf}{\lim\limits_{A \to \infty}}
\newcommand{\Rtinf}{\lim\limits_{R \to \infty}}

\newcommand{\ntx}[1]{\lim\limits_{n \to #1}}
\newcommand{\xtx}[1]{\lim\limits_{x \to #1}}
\newcommand{\ttx}[1]{\lim\limits_{t \to #1}} 
\newcommand{\ktx}[1]{\lim\limits_{k \to #1}} 
\newcommand{\dxtx}[1]{\lim\limits_{\Delta x \to #1}}

\newcommand{\jfab}{\int_{a}^{b}}
\newcommand{\jf}[2]{\int_{#1}^{#2}}

\newcommand{\nsum}[2]{\sum\limits_{n=#1}^{#2}}
\newcommand{\isum}[2]{\sum\limits_{i=#1}^{#2}}
\newcommand{\ksum}[2]{\sum\limits_{k=#1}^{#2}}

\newcommand{\nsuminf} {\nsum{1}{\infty}}
\newcommand{\ksuminf} {\ksum{1}{\infty}}
\newcommand{\isuminf} {\isum{1}{\infty}}



$\nl$

\begin{center}第6章 不定积分  \end{center}



$\S 1. \  一、原函数$

$证明
f(x) = \begin{cases} 0, & x \ne 1 \\ 1, & x=0 \end{cases}
在R无原函数,但可积
$

$反证法:设F'(x)=f(x),则F(x) \in C(R)$

$故
F(x) = \begin{cases} C_1, & x>0  \\ C_2, & x<0 \end{cases}
且C_1=C_2,故F(0)=C_1,矛盾
$

导函数无首类间断点(导数介值定理)


$定理1.f(x) \in C(x)则必有原函数F(x) (充分条件)$

$例:\int_{}^{} (2x+1)^{100}\, dx = \frac{1}{2} \frac{(2x+1)^{101}}{101}+c$

$法则1、\sum 法$

$法则2、凑法。例:$

$\int \frac{1}{a^2+x^2}\,dx=\frac{1}{a} arctg\frac{x}{a}+c$

$\int \frac{dx}{x^2+2x+3}\,dx=\frac{1}{\sqrt 2}arctg \frac{x+1}{\sqrt 2}+c$

$\int \frac{x}{x^2+2x+3}\,dx =\frac{1}{2} \int \frac{2x+2}{x^2+2x+3}\,dx-\int \frac{1}{x^2+2x+3}\,dx$

$=\frac{1}{2}ln(x^2+2x+3)-\frac{1}{\sqrt 2}arctg \frac{x+1}{\sqrt 2}+c$

$\nl$

$例:\jf{}{} max(1,x^2)dx$

$=F(x)+C=\begin{cases} \frac{1}{3}x^3+C_1,x<-1 \\ x+C,|x|\le 1 \\ \frac{1}{3}x^3+C_2,x>1 \end{cases}$

$故C_1=-\frac{2}{3},C_2=\frac{2}{3}$

$法则3、‘换元’法$

$例:\int \frac{e^x(1+e^x)}{\sqrt{1-e^{2x}}}令e^x=sint$

$倒变换x=\frac{1}{t}$

若分母次数高于分子时,利用综合除法

$遇形如1+x^2的情况可以令x=tgt等,若在分母中可令其=t^2$

$法则4、‘分部’法,反对幂指三,越在后越先积$

$反例:\int \frac{xe^x}{(1+x)^2}\,dx = \int xe^x d(\frac{-1}{1+x})=\frac{-x}{1+x}e^x+\int \frac{1}{1+x}d(xe^x)$

$\nl$
$递推公式$

$Z_n=\int sec^nxdx=\int sec^{n-2}d(tgx)$

$=sec^{n-2}tgx-(n-2)\int sec^{n-2}xtg^2xdx$

$=sec^{n-2}tgx-(n-2)(Z_n-Z_{n-2})$

$故Z_n=....$

$\nl$

$I_n=\int \frac{dx}{x^n \sqrt{1+x^2}}$

$=\int \frac{(1+x^2)-x^2}{x^n \sqrt{1+x^2}}dx$

$=\int \frac{\sqrt{1+x^2}}{x^n}dx-\int \frac{1}{x^{n-2} \sqrt{1+x^2}}dx$

$=\frac{1}{1-n} \sqrt{x^2+1}d(x^{1-n})-I_{n-2}$

$=\frac{1}{1-n}x^{1-n}\sqrt{x^2+1}-\frac{1}{1-n}\int x^{1-n}\frac{x}{\sqrt{1+x^2}}dx-I_{n-2}$

$=\frac{1}{1-n}x^{1-n}\sqrt{x^2+1}-\frac{1}{1-n}I_{n-2}-I_2$

$=\frac{1}{1-n}x^{1-n}\sqrt{x^2+1}-\frac{2-n}{1-n}I_{n-2}$

$\nl$

$I_n=\int \frac{dx}{(x^2+a^2)^m}=\frac{x}{(x^2+a^2)^m}+2mH_m-2ma^2H_{m+1}$

$I_n=\int \frac{dx}{(x^2+px+q)^m}=\int \frac{d(x+\frac{p}{2})}{((x+\frac{p}{2})^2+..)^m}$

$I_m=\int \frac {Ax+B}{(x^2+px+q)^m}dx=\frac{A}{2}\int \frac{2x+pdx}{...}+\int \frac{B-\frac{Ap}{2}}{...}$

$\int \frac{a_1sinx+b_1cosx}{asinx+bcosx}dx=Ax-Bln|asinx+bcosx|+C$

$A=\frac{aa_1+bb_1}{a^2+b^2},B=\frac{a_1b-b_1a}{a^2+b^2}$

$\nl$

积分计算中常用到sgnx,但最好用讨论消去

$\nl$

$Eular变换(灵活观察,选择合适)$

$\nl$
$(1)a>0,令\sqrt {ax^2+bx+c}=\pm \sqrt a x+t(或\sqrt a x \pm t)$

$bx+c=\pm 2 \sqrt a tx+t^2 \Rightarrow x=\frac{t^2-c}{b \mp 2\sqrt a t}$
$\nl$

$设G=\{(\alpha_x,\beta_x)|x \in (a,b)\}为(a,b)的开覆盖$

$证明:\exists l>0,使得对\forall x \in [a,b] \cup (x,l)均被G中某区间覆盖$

$\nl$
$(2)若c>0,可令\sqrt{ax^2+bx+c}=xt\pm \sqrt c$

$有ax+b=xt^2 \pm 2\sqrt c t \Rightarrow x=\frac{b \mp 2 \sqrt c t}{t^2-a}$

$\nl$
$(3)若\sqrt{ax^2+bx+c}有实根,即原式=\sqrt{a(x-\alpha)(x-\beta)}$

$令\sqrt{ax^2+bx+c}=t(x-\alpha)或t(x-\beta)$

$a(x-\beta)=t^2(x-\alpha)即x=\frac{\alpha t^2-\alpha \beta}{t^2-\alpha}$

$\nl$

$\int \frac{dx}{x\sqrt {4-x^2}}一题多解$

$①x=2sint$

$②x^2=t$

$③\sqrt{4-x^2}=t$

$④x=\frac{1}{t}$

$⑤\sqrt{\frac{2+x}{2-x}}=t$

$⑥x^2=\frac{1}{t}$

$⑦令\sqrt{4-x^2}=tx \to ...=\frac{1}{2}\int \frac{dt}{\sqrt{1+t^2}}$

$⑧\sqrt{4-x^2}=tx+2 \to ....=\frac{1}{2} \int \frac{1}{t}dt$

$⑨令\sqrt{4-x^2}=t(2-x) \to ...=\frac{1}{2}\int \frac{dt}{t^2}$






\end{document}

